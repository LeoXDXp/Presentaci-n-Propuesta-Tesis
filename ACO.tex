\section{Hormigas}
%\subsection{Metaheur\'istica ACO }
\begin{frame}{Hormigas (OE 1)}
    \centering
    \includegraphics[height=2.5in]{Pictures/ACO-ant.png}
\end{frame}

\note[itemize]{
\small
    \item Como se mencion\'o previamente, uno de los problemas principales al individualizar filamentos es que se desconoce el comienzo y el final de un filamento al comienzo del problema. 
    \item As\'i, resulta natural explorar el comportamiento de las hormigas al buscar alimento como inspiraci\'on para individualizar filamentos. Este comportamiento se puede resumir en las 3 etapas que se muestran en la imagen.
    \item En una primera etapa, a la izquierda de la imagen en la letra {\bf a}, se puede observar que una hormiga que sale del hormiguero recorre un camino aleatorio hasta llegar al alimento. En su retorno, por el mismo camino, deposita feromonas para indicarle a otras hormigas por donde fue su recorrido.
    
    \item Luego, otras hormigas pueden tomar esta informaci\'on en su propio descubrimiento de un camino hacia el alimento, como se observa el la mitad de la imagen, en {\bf b}
    
    \item Finalmente, a la derecha de la imagen, se observa que las hormigas depositan m\'as y m\'as feromonas en el camino m\'as corto entre el hormiguero y el alimento, causando una convergencia sobre un camino \'optimo. Esta convergencia es lo que se define como el ciclo de reforzamiento positivo

    \item La comunicaci\'on indirecta entre las hormigas que conduce a la convergencia sobre un camino \'optimo se describe como el modelo de feromonas
}



\begin{frame}{Modelo de Feromonas para la Individualizaci\'on de Filamentos (OE 1)}
\small
\begin{itemize}
    \item Comunicaci\'on indirecta mediante feromonas
    \item B\'usqueda de una o m\'as soluciones
    \item Recorrido aleatorio
    \item Representaci\'on de filamentos mediante un grafo
\end{itemize}
    \begin{algorithm}[H]
    \SetAlgoLined
     Ajuste de Par\'ametros \& inicializaci\'on de feromonas\;
     \While{Criterio de finalizaci\'on no se cumple}{
       Construcci\'on\_de\_soluci\'on\_de\_cada\_hormiga()\;
       M\'etodo\_de\_b\'usqueda\_no\_local(); //Paso opcional\\
       Actualizaci\'on\_de\_feromonas()\;
     }
     \caption{Algoritmo metaheur\'istica ACO}\label{ACO-Algo}
    \end{algorithm}
\end{frame}

\note[itemize]{
    \item El modelo de feromonas sirve de inspiraci\'on para la metaheur\'istica ACO, mediante la cual es posible encontrar no solo una \'unica soluci\'on, sino que tambi\'en un conjunto de soluciones.
    
    \item  La representaci\'on de un conjunto de caminos en ACO puede ser interpretado como un grafo, donde cada camino corresponde a un filamento. 
    \item En si, la metaheur\'istica ACO cuenta con 4 etapas, 3 de las cuales se encuentran en un ciclo y cuyo orden puede variar con respecto a lo que se presenta en el algoritmo 1. Esta flexibilidad es una de las caracter\'isticas de la metaheur\'istica.
    \item Estas etapas son la construcci\'n de la soluci\'on por parte de cada hormiga, la ejecuci\'on de un m\'etodo de b\'usqueda no local, y la actualizaci\'on de las feromonas en el camino recorrido.
}

\begin{frame}{Etapas del Algoritmo Propuesto (OE 1)}
\centering  
\includegraphics[height=2.8in]{Pictures/ACOdiagram.png}
\end{frame}

\note[itemize]{
    \item En particular,  el uso de ACO para la individualizaci\'on de filamentos se refleja en el diagrama, siendo ACO la parte central del algoritmo propuesto.
    \item Para preparar los datos de entrada para ACO, es necesario extraer desde una imagen un grafo que represente una red de filamentos, utilizando alguna herramienta externa.
    \item Adicionalmente a la extracci\'on, es posible alimentar algunos par\'ametros de entrada de ACO en el algoritmo propuesto mediante informaci\'on conocida a priori del comportamiento de la c\'elula observada en la imagen de la que se extrae el grafo.
    \item Con lo anterior, el algoritmo propuesto entra al ciclo donde se encuentran las 3 etapas descrita en el algoritmo 1: La construcci\'on de caminos, la actualizaci\'on de feromonas y el m\'etodo de b\'usqueda no local
}

%\subsection{Problema de Satisfacci\'on de Restricciones}
% \begin{frame}{ACO aplicado como Problema de Satisfacci\'on de Restricciones (OE 1)}
%   \begin{columns}
%     \begin{column}{0.5\textwidth}
%         \begin{itemize}
%           \item $P = (S, \Omega, F)$
%           % esta definido por un conjunto discreto de variables
%           \item S:  Cjto. de soluciones, compuesto por variables discretas $X_{i}$, $i \in 1 \dotsc n$
%           %\item $S:\quad X = v_{i}^{j} \in D_{i} = \{v_{i}^{1} \dotsc  v_{i}^{|D_{i}|}\}$.
%           \item Sol. Candidata $s \in S$, y $s^{*}$ una soluci\'on \'optima
%           \item Cjto. de Restricciones $\Omega$
%           \item Funci\'on objetivo $F: S\rightarrow \mathbb R_{0}^{+}$
%           %\item  
          
%       \end{itemize}
%     \end{column}
%     \begin{column}{0.5\textwidth}
%       \begin{itemize}
%           \item $s$ es factible si satisface las restricciones de $\Omega$
%           % y se relacionan mediante 
%           \item Pueden existir m\'ultiples soluciones $s^{*}$
%           \item $S^{*}$ es el conjunto que engloba todas las soluciones \'optimas
%           \item $s^{*} \in S^{*} \subseteq S$
%       \end{itemize}
%     \end{column}
% \end{columns}
% \end{frame}





% \begin{frame}{Adaptaci\'on a Individualizaci\'on de Filamentos}
%     \begin{algorithm}[H]
%     \SetAlgoLined
%     \KwData{Variables $X_i \dotsc X_n$, dominios $D_1 \dotsc D_n$, Restricciones $\in \Omega$}
%     \KwResult{conjunto s\textquotesingle $ \subseteq S$ != $\emptyset$, si existen soluciones factibles}
%      Ajuste de Par\'ametros \& inicializaci\'on de feromonas \;
%      \While{Criterio de finalización no se cumple}{
%       Construcci\'on\_de\_soluci\'on\_de\_cada\_hormiga()\;
%       M\'etodo\_de\_b\'usqueda\_no\_local(); //Paso opcional\\
%       Actualizaci\'on\_de\_feromonas()\;
%      }
%      \caption{Algoritmo de un modelo COP adaptado a una metaheur\'istica ACO}\label{COP-ACO-Algo}
%     \end{algorithm}
% \end{frame}


\section{Algoritmo Propuesto}

\begin{frame}{ACO: Construcci\'on de Soluci\'on, Asignaci\'on de 1ra Arista (OE 1)}
    \begin{columns}
        \begin{column}{0.43\textwidth}
             Heur\'istica de Asignaci\'on Inicial: \begin{enumerate}
                    \item Arista con un nodo grado 1
                    \item Aristas con un nodo en intersecciones
                    \begin{itemize}
                        \item \'Angulo en [$\theta$, $Max\_Angle$]
                    \end{itemize}
                    \item Arista aleatoria
                \end{enumerate}
                
            
        \end{column}
        \begin{column}{0.57\textwidth}
            \centering
            \includegraphics[scale=0.45]{Pictures/ant-initial-edge.png}
        \end{column}
    \end{columns}
\end{frame}

%\subsection{Construcci\'on de Soluciones}
% \begin{frame}{ACO: Construcci\'on de Soluci\'on (OE 1)}
% \centering
% \includegraphics[scale=0.4]{Pictures/ACO-ant-Constr.png}
% \begin{itemize}
%     \item Influencia de feromonas $\tau$ y de una heur\'istica $\eta$ en la elecci\'on de Aristas/Componentes $c_i$ que respeten restricciones en $\Omega$
%     \item Aspecto aleatorio en la elecci\'on
% \end{itemize}
% \end{frame}

% \begin{frame}{ACO: Heur\'istica de Asignaci\'on Inicial}
% \begin{enumerate}
% \item Arista $a: (n_i,n_j)$ tal que $deg(n_i) = 1$ o $deg(n_j) = 1$ 
% %La arista a asignar debe tener al menos uno de sus nodos con grado 1, indicando que es el inicio o final de una parte del grafo.

% \item De no haber aristas disponibles con esas caracter\'isticas, se realiza una asignaci\'on inicial de una arista que cumpla con 2 criterios:
% \begin{itemize}
%     \item $a: (n_i,n_j)$ tal que $deg(n_i) >= 2$ o $deg(n_j) >= 2$ 
%     \item El \'angulo que forma la arista candidata a elegir, junto a otra arista a la que pertenece el nodo, debe pertenecer al rango $]\theta, Max\_Angle]$.
% \end{itemize}

% \item Una arista aleatoria que no pertenezca a una soluci\'on o camino evaluado como de buena calidad
% %. La calidad de un camino se presenta m\'as adelante en esta secci\'on.
% \end{enumerate}
% \end{frame}



\note[itemize]{
    \small
    \item la etapa de construcci\'on de una soluci\'on se compone de 2 partes: Primero se debe asignar la 1ra arista por donde la hormiga comenzar\'a su recorrido, para luego comenzar el proceso iterativo de elegir la siguiente arista a recorrer.
    \item La asignaci\'on de la primera arista se realiza mediante una heru\'istica que consta de 3 criterios. El enfoque de estos criterios se centra en la exploraci\'on del espacio de b\'usqueda
    \item El primer criterio busca asignar aristas que esten en el borde del grafo, como la arista con uno de sus nodos en A1.
    
    \item De no existir m\'as aristas como las del 1er criterio, se buscan aristas que cumplan con 2 condiciones: 1ro, que se encuentren en intersecciones, es decir que tengan uno de sus nodos con grado 2 o superior, como el nodo azul de la arista A2 y 2do, que el \'angulo que forma con otras aristas conectadas mediante el nodo con grado 2 o superior, se encuentre en el rango definido por 2 par\'ametros, theta y Max Angle. Esto debido a que en varios tipos de filamentos hay casos en que los filamentos no nacen s\'olo del borde, sino tambi\'en a partir de otros filamentos.  
    
    \item De no contar con aristas disponibles seg\'un los criterios previos, se elige una arista al azar, privilegiando las aristas que no son parte de un camino de otra hormiga que ya fue evaluado y calificado como una soluci\'on factible.
    
   \item Una vez que se le asigna una arista inicial a la hormiga, esta debe elegir que nuevas aristas irá agregando en su camino 

}


\begin{frame}{ACO: Construcci\'on de Soluci\'on (OE 1)}
    \begin{columns}
        \begin{column}{0.65\textwidth}
            \centering
            \includegraphics[scale=0.4]{Pictures/ACO-ant-Constr-choices.png}
        \end{column}
        \begin{column}{0.35\textwidth}
        \small
            \begin{equation}
            P(c_{i} | s^{P}) = \frac
            {\tau_{i} ~ \eta_{i}}
            {\sum\limits_{c_{j}\in N(s^p)}{\tau_{j} ~ \eta_{j} } } %, \forall c_{i} \in N(s^{P}).
            \label{eq:antProbabilities}
            \end{equation}
        \end{column}
    \end{columns}

    \begin{columns}
        \begin{column}{0.35\textwidth}
            \begin{itemize}
                \item Calidad $\sim$ Funci\'on Objetivo 
                %\item Calidad $s^{P}_1$ = $\sum \eta_{i}$
                \item Calidad $s_1$ = $\frac{1}{n}\sum \eta_{i} $
            \end{itemize}
        \end{column}
        \begin{column}{0.65\textwidth}
            \begin{itemize}%\fontsize{9pt}{10}\selectfont
                \item Calidad M\'inima $\geq \frac{Max\_Score}{2}$
                \item Buena Calidad: $Max\_Score$
                \item Calidad Intermedia: $[\frac{Max\_Score}{2}, Max\_Score[$
            \end{itemize}
        \end{column}
    \end{columns}
\end{frame}

\note[itemize]{
   \small
    \item Cada una de las posibles aristas vecinas a elegir tiene una probabilidad de ser elegida. Esta probabilidad se define seg\'un la ecuaci\'on a la derecha de la diapositiva y en esta influye la feromona $\tau$ y el valor de una heur\'istica $\eta$ (eta) asociada a cada arista.
    \item Esta heur\'istica entrega una puntuaci\'on asociada al \'angulo que forma la arista vecina con la \'ultima arista que fue agregada al camino o soluci\'on.
    
    \item a modo de ejemplo, las aristas en rojo representan las aristas vecina de C6, que fue la ùltima en ser a\~nadida hasta ese punto al recorrido de la hormiga, teniendo cada una de las aristas rojas una probabilidad de ser elegida. 

    \item Una vez que la hormiga termina su recorrido, se evalua la soluci\'on promediando los valores de eta de las aristas elegidas. Esta evaluaci\'on corresponde a la funci\'on objetivo de la metaheur\'istica ACO y se puede separar en 3 rangos: 
    
    \item Las soluciones que no cumplen con una calidad m\'inima, que se desechan
    \item Las soluciones que se definen como de buena calidad, que son las que se eligen como soluciones candidatas
    \item y las soluciones que requieren de mayor evaluaci\'on dado que con los criterios aplicados durante la construcci\'on no pueden ser desechados ni aceptados.
    \item Como se observa en la presentaci\'on, el par\'ametro Max Score es el que define los rangos de calidad.
    
    \item (Respirar) Luego, las soluciones de calidad intermedia y superior pasan a las siguientes etapas, dado que se busca maximizar la funci\'on objetivo
}

%\subsection{M\'etodo de b\'usqueda no local}
\begin{frame}{B\'usqueda no local: l\'ogicas globales/centralizadas (OE 1)}
    
    \begin{columns}
        \begin{column}{0.4\textwidth}
            \begin{itemize}
                \item Eliminar soluciones candidatas que no aporten informaci\'on nueva
                \item Soluciones parcialmente repetidas
            \end{itemize}
        \end{column}
        \begin{column}{0.2\textwidth}
        \includegraphics[scale=0.5]{Pictures/ant-segments-repetead-sol1.png}
        \end{column}
        \begin{column}{0.2\textwidth}
        \includegraphics[scale=0.5]{Pictures/NoConsenso2.png}
        \end{column}
        \begin{column}{0.2\textwidth}
        \includegraphics[scale=0.5]{Pictures/NoConsenso3.png}
        \vspace{0.5cm}
        \includegraphics[scale=0.5]{Pictures/NoConsenso4.png}
        \end{column}
    \end{columns}
\end{frame}

\note[itemize]{
\small
    \item Posterior a la construci\'on de soluciones, la siguiente etapa corresponde a la b\'usqueda no local, la que se basa en l\'ogicas globales o centralizadas, a diferencia de la construcci\'on de soluciones de cada hormiga que es miope.
    %\item La b\'usqueda de soluciones de cada hormiga puede llevar a que dos o m\'as hormigas encuentren soluciones que son muy similares, repitiendo informaci\'on.
    \item Esta etapa busca eliminar soluciones duplicadas o muy similares, privilegiando las soluciones que contengan dentro de si a otras. A modo de ejemplo, en naranjo se muestran dos soluciones similares, siendo la soluci\'on de la derecha levemente m\'as larga.
    \item Se elige la soluci\'on m\'as larga para evitar ambiguedades y que en caso de error, la soluci\'on indicada como un filamento sea igual o m\'as extensa en vez de ser m\'as corta con respecto a lo que pueda indicar un experto
    \item Este an\'alisis es similar al que puede suceder entre 2 expertos que ante la misma imagen identifiquen distintos filamentos, como se ejemplifica con las im\'agenes de la derecha,  y que se explic\'o previamente.
    
}

% \begin{frame}{ACO: Actualizaci\'on de Feromonas (OE 1)}
% \centering
% \includegraphics[scale=0.4]{Pictures/ACO-ant-ferom.png}
% \begin{itemize}
%     
% \end{itemize}
% \end{frame}


%\subsection{Anti-feromonas}
\begin{frame}{ACO: Actualizaci\'on de feronomas, uso de Anti-feromonas SAP (OE 1)}
    \centering
    \includegraphics[scale=0.4]{Pictures/ACO-ant-ferom-penalize.png}
    \begin{columns}
        \begin{column}{0.5\textwidth}
            \begin{itemize}
            \small
                \item Evaluaci\'on de factiblidad de soluciones candidatas $s \in S$
                \item Feronomonas: premiar los caminos de buena calidad para lograr onvergencia sobre un camino \'optimo
                
            \end{itemize}
        \end{column}
        \begin{column}{0.5\textwidth}
            \begin{itemize}
            \small
                \item Cambio: $\tau_0 \longrightarrow \gamma$
                \item $\gamma$ es un factor penaliza los caminos de mala calidad
                \item 2 penalizaciones $\longrightarrow \tau_i = 0$ 
            \end{itemize}
        \end{column}
    \end{columns}
    
\end{frame}

\note[itemize]{
    \item la \'ultima etapa a mencionar corresponde a la actualizaci\'on de feromonas. Al finalizar el recorrido de una hormiga, es necesario actualizar el valor de la feromonas. En el caso de una metaheur\'istica ACO tradicional, se aumenta el valor de tau para cada arista que pertenece a un camino de buena calidad.
    \item Sin embargo, ese enfoque privilegia la convergencia sobre un sola soluci\'on, y a su vez puede ocasionar la perdida de soluciones factibles.
    \item Para evitar eso, se propone el uso de Anti-feromonas, cuyo objetivo es acotar el espacio de b\'usqueda mediante la penalizaci\'on de los caminos o soluciones de mala o baja calidad.
    \item As\'i, el valor de tau que influye en la probabilidad de elecci\'on de una arista disminuye, pudiendo llegar a cero, lo que ocasiona que esa arista no pueda ser elegida por ninguna hormiga que aun este construyendo una soluci\'on.
}

\begin{frame}{Problema de Anti-feromonas SAP (OE 1)}
\centering
\includegraphics[scale=0.4]{Pictures/ACO-ant-constr-penalize.png}
    \begin{itemize}
        \item Problema: penalizaci\'on puede ocasionar perdida de soluciones factibles
        \item Propuesta: Relacionar penalizaci\'on de $\tau_i$ con $c_i \in s^P$
    \end{itemize}
\end{frame}

\note[itemize]{
    \item Sin embargo, el cambio de feromonas a anti-feromonas no soluciona el problema que se puede originar al focalizar la penalizaci\'on solo en las aristas.
    
    \item Una arista con un tau penalizado por pertenecer a uno o m\'as caminos de mala calidad, puede bloquear otros caminos, dado que el valor de tau no guarda relac<i\'on con las otras aristas que conforman el camino de mala calidad que caus\'o la penalización. As\'i, las soluciones que pudiesen pasar por esa arista se ven limitadas.
    
    \item A modo de ejemplo, el camino verde que es de mala calidad origina una penalizaci\'on sobre la arista en rojo. Luego, el camino azul, que puede construir una soluci\'on de buena calidad si incorpora la arista en rojo, se ve bloqueado, ya que la arista roja tiene una penalizaci\'on que hace menos probable o derechamente imposible su elecci\'on.
    
    

}

\begin{frame}{Anti-feromonas SAP dependientes del camino previo (OE 1)}
    \centering
    \includegraphics[scale=0.51]{Pictures/ACO-ant-ferom-penalize-seg.png}
\end{frame}

\note[itemize]{
    \item Por ende, se propone un cambio para que la penalizaci\'on de una arista guarde relaci\'on con el conjunto de aristas que la preceden en el camino
    \item As\'i, solo hormigas que repitan parcial o totalmente el camino verde no podran elegir la arista en rojo, mientras que hormigas que provengan de otro recorrido, como la hormiga del camino azul no se ver\'an afectadas, evitando la perdida de soluciones.
    
    %\item El conjunto de aristas predecesoras se define como un segmento.

}

% \begin{frame}{Anti-feromonas sobre un segmento de una sola arista (OE 1)}
%     \centering
%     \includegraphics[height=2in]{Pictures/ant_segments_complex_case_B2.png}
%     \hspace{0.1cm}
%     \includegraphics[height=2in]{Pictures/ant_segments_complex_case_B_blocked.png}
% \begin{itemize}
%     \item Segmentos de una sola arista ocasionan mismo problema que se intenta resolver
% \end{itemize}
% \end{frame}

% \begin{frame}{Anti-feromonas sobre un segmento (OE 1)}
% \centering
%     \includegraphics[scale=0.5]{Pictures/ant_segments_complex_case_B2_extended.png}
%     \begin{itemize}
%         \item Se extiende el segmento unitario con el segmento que lo precede
%     \end{itemize}
% \end{frame}

%\subsection{Criterios para la Actualizaci\'on de Anti-feromonas}
\begin{frame}{Criterios de Anti-feromonas sobre soluciones de calidad intermedia (OE 1)}
\begin{itemize}
    \item Hormigas con soluciones de calidad inferior al m\'inimo de $\frac{Max\_Score}{2}$ han sido desechadas
    \item Hormigas con soluciones de buena calidad ($calidad = Max\_Score$) ya han sido aceptadas
\end{itemize}
Criterios de anti-feromonas para calidad intermedia:
\begin{enumerate}
    \item Curvatura de una soluci\'on
    \item Magnitud de desplazamiento entre segmentos
    \item Criterio espec\'ifico para neuronas
\end{enumerate}
\end{frame}

\note[itemize]{
    \item La anti-feromona se aplica sobre hormigas que al finalizar su recorrido entregan una soluci\'on cuya calidad sea intermedia, dado que las hormigas con soluciones cuya calidad es inferior a intermedia ya han sido desechadas, mientras que las de buena calidad ya han sido seleccionadas como soluciones factibles.
    \item Luego, de los 3 criterios utilizados para actualizar la penalizaci\'on mediante las anti-feromonas,  los 2 primeros se relacionan a la rigidez que presenta un filamento, el que var\'ia dependiendo de la estructura observada, mientras que el 3ro, que aplica solo a las neuronas, busca validar un comportamiento espec\'ifico, necesario para aceptar una soluci\'on.
    
    \item NO LEER: Solo si se pregunta: Las soluciones menores a la calidad mínima son las que tienen los angulos entre sus aristas mayores a max angle
}

\begin{frame}{Curvatura de una soluci\'on (OE 1)}
\centering
    \includegraphics[scale=0.5]{Pictures/ant_curvature_case.png}
    \begin{itemize}
        \item Se definen $\Vec{p} = v^{a}_1 - mc^{a}$ y $\Vec{q} = mc^{a} - v^{a}_n$.
        \item El \'angulo entre la proyecci\'on de $\Vec{p}$ y $\Vec{q}$ debe ser menor a $\theta \times Max\_Axial\_Displacement$ para no descartar la soluci\'on.
        \end{itemize}
\end{frame}

\note[itemize]{
\item Con respecto al primer criterio, el c\'alculo de la curvatura corresponde al \'angulo formado entre el vector q y la proyecci\'on del vector p. El vector q se construye con el nodo inicial del recorrido, v1 en el ejemplo y el centro de masa, mientras que el vector q lo hace con el centro de masa y el nodo final del recorrido, vn.

\item Este \'angulo debe respetar el umbral que define la multiplicaci\'on de los par\'ametros theta y Max axial displacement
\item una soluci\'on como la destacada en color naranja se puede calificar como una soluci\'on infactible debido a su pronunciada curvatura, lo que no es normal de observar en los filamentos. 

}

\begin{frame}{Magnitud de desplazamiento entre segmentos (OE 1)}
    \centering
    \includegraphics[height=2in]{Pictures/ant_segmentMagnitude_case.png}
    \hspace{0.2cm}
    \includegraphics[height=2in]{Pictures/ant_segmentMagnitude_case_2.png}
    \begin{itemize}
        %\item El criterio de curvatura puede no ser suficiente por si mismo
        \item Comportamiento esperado de la rigidez permite descartar soluciones con cambios demasiado pronunciados entre sus segmentos
    \end{itemize}
\end{frame}

\note[itemize]{
\small
\item Para explicar el segundo criterio, se hace necesario aclarar a que corresponde un segmento. Un segmento esta formado por una o m\'as aristas, donde cada arista del segmento forma un \'angulo en el rango $[0, \theta]$ con sus vecinos. Un ejemplo de un segmento es el formado por las aristas e4 y e7 en la parte superior del camino indicado en naranja. Las aristas que tienen \'angulos mayores a theta con sus vecinos conforman segmentos de una sola arista.

\item Una ventaja de utilizar segmentos de camino es evitar volver a evaluar la relaci\'on entre todas las aristas que conforman el camino, ya que al tener aristas que forman \'angulos en el rango $[0, \theta]$, podemos asumir que forman parte del mismo filamento.

\item Luego, a partir de cada par de segmentos vecinos en una soluci\'on, se  selecciona el segmento de mayor longitud, el que servir\'a para determinar cual es el movimiento proyectado con respecto al movimiento reflejado por el segmento de menor longitud.

\item En este ejemplo, podemos observar una soluci\'on que cumple con el criterio de curvatura, pero que presenta una diferencia marcada entre los segmentos 1 y 2. 

\item Existe una tolerancia de cuanto pueden variar los segmentos entre s\'i. En el algoritmo propuesto se define esta tolerancia considerando el par\'ametro Max Axial Displacement as\'i como la magnitud de los segmentos comparados.

\item Esto permite descartar soluciones que respeten el criterio de curvatura, pero que presentan movimientos pronunciados o muy bruscos que sobrepasan la rigidez esperada de un filamento.
}

%%\subsection{Extracci\'on de informaci\'on para individualizar filamentos}
\begin{frame}{Criterio espec\'ifico para neuronas (OE 1)}
    \begin{itemize}
        \item Se debe validar que los filamentos de una neurona parten del {\bf soma}
        \item  Extracci\'on de informaci\'on: caracter\'isticas geom\'etricas, topol\'ogicas y espaciales
    \end{itemize}
    \begin{columns}
        \begin{column}{0.55\textwidth}
        \centering
        \includegraphics[height=1.7in]{Pictures/Porta183-somaEdges-example2.png}
        \end{column}
        \begin{column}{0.45\textwidth}
            \begin{equation}
                \tau_{ij} \leftarrow
                    \begin{cases}
                     \tau_{ij} \cdot \gamma \text{ si } deg(v^{a}_{n}) = 1,  \\[3ex]
                    
                    \text{0 si } \tau_{ij} \leq 0.25, \\[3ex]
                    \tau_{ij} \quad \text{en otro caso}.
                    \end{cases}
            \end{equation}
        \end{column}
    \end{columns}
    
\end{frame}

\note[itemize]{
  \item Si la c\'elula observada es una neurona, es necesario validar que los filamentos de una neurona parten del {\tt soma} o centro de la misma, y que filamentos posteriores que no comiencen del soma solo pueden nacer a partir de otros filamentos. Un camino que termine en un nodo final de grado de 1 \textbf{no} respeta aquel comportamiento, debido a que en un grafo que representa una red de filamentos de una neurona, los nodos ubicados en el soma o en la intersecci\'on entre filamentos presentan un grado mayor a 1.
  
  \item La identificaci\'on de los nodos del soma se puede realizar de forma aproximada mediante algoritmos de grafos como closeness centrality. As\'i mismo, algoritmos aplicables a grafos pueden permitir extender la extracci\'on de informaci\'on para agregar nuevos criterios de anti-feromonas para casos espec\'ificos.
  
  \item As\'i, se utiliza el grado del nodo final para penalizar mediante la anti-feromona.
}
% \begin{frame}{Extracci\'on de informaci\'on para individualizar filamentos}
%      \begin{figure*}[h!]
%     \centering
%     \begin{subfigure}[t]{0.48\textwidth}
%         \centering
%         %\includegraphics[height=1.5in]{Pictures/50-ROIs-Spinning-Marchantia-somaEdges.png}
%     \end{subfigure}%
%     ~ \hspace{0.1cm}
%     \begin{subfigure}[t]{0.48\textwidth}
%     \centering
%         \includegraphics[height=1.5in]{Pictures/Porta18-3a1-somaEdges.png}
%     \end{subfigure}
%     \end{figure*}
%     \begin{itemize}
%         \item Caracter\'isticas geom\'etricas, topol\'ogicas y espaciales
%     \end{itemize}
% \end{frame}