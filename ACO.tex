\section{Hormigas}
\subsection{Metaheur\'istica ACO }
\begin{frame}{Hormigas (Obj. Esp. 1)}
\includegraphics[width=0.9\textwidth]{Pictures/ACO-ant.png}
\end{frame}

\begin{frame}{Hormigas}
\centering
\includegraphics[width=0.9\textwidth]{Pictures/ACO-ant-2.png}
\end{frame}

\begin{frame}{Modelo de Feromonas}
\small
\begin{itemize}
    \item Comunicaci\'on indirecta mediante feromonas
    \item Recorrido aleatorio
    \item Convergencia sobre el camino m\'as corto
    \item Opciones de caminos posibles representan el espacio de b\'usqueda
    \item Orden no estricto
\end{itemize}
    \begin{algorithm}[H]
    \SetAlgoLined
     Ajuste de Par\'ametros \& inicializaci\'on de feromonas\;
     \While{Criterio de finalizaci\'on no se cumple}{
       Construcci\'on\_de\_soluci\'on\_de\_cada\_hormiga()\;
       M\'etodo\_de\_b\'usqueda\_no\_local(); //Paso opcional\\
       Actualizaci\'on\_de\_feromonas()\;
     }
     \caption{Algoritmo metaheur\'istica ACO}\label{ACO-Algo}
    \end{algorithm}
\end{frame}

\begin{frame}{Adaptaci\'on de ACO a un problema de satisfacci\'on de restricciones}
    \begin{itemize}
        \item Hormigas Sint\'eticas o Agentes
        %\item Basado en el comportamiento de las hormigas reales
        \item Entendiendo los caminos posibles a recorrer por las hormigas como un grafo
        \item Camino $s$ compuesto por aristas o componentes de soluci\'on $c_i$
        \item Valor de la feromona depositada $\tau_{i}$
        \item Se pueden agregar restricciones $\Omega$ en el desplazamiento
    \end{itemize}
\end{frame}

\subsection{Problema de Satisfacci\'on de Restricciones}
\begin{frame}{ACO aplicado como Problema de Satisfacci\'on de Restricciones}
  \begin{columns}
    \begin{column}{0.5\textwidth}
        \begin{itemize}
          \item $P = (S, \Omega, F)$
          % esta definido por un conjunto discreto de variables
          \item S:  Cjto. de soluciones, compuesto por variables discretas $X_{i}$, $i \in 1 \dotsc n$
          %\item $S:\quad X = v_{i}^{j} \in D_{i} = \{v_{i}^{1} \dotsc  v_{i}^{|D_{i}|}\}$.
          \item Sol. Candidata $s \in S$, y $s^{*}$ una soluci\'on \'optima
          \item Cjto. de Restricciones $\Omega$
          \item Funci\'on objetivo $F: S\rightarrow \mathbb R_{0}^{+}$
          %\item  
          
      \end{itemize}
    \end{column}
    \begin{column}{0.5\textwidth}
      \begin{itemize}
          \item $s$ es factible si satisface las restricciones de $\Omega$
          % y se relacionan mediante 
          \item Pueden existir m\'ultiples soluciones $s^{*}$
          \item $S^{*}$ es el conjunto que engloba todas las soluciones \'optimas
          \item $s^{*} \in S^{*} \subseteq S$
      \end{itemize}
    \end{column}
\end{columns}
\end{frame}





% \begin{frame}{Adaptaci\'on a Individualizaci\'on de Filamentos}
%     \begin{algorithm}[H]
%     \SetAlgoLined
%     \KwData{Variables $X_i \dotsc X_n$, dominios $D_1 \dotsc D_n$, Restricciones $\in \Omega$}
%     \KwResult{conjunto s\textquotesingle $ \subseteq S$ != $\emptyset$, si existen soluciones factibles}
%      Ajuste de Par\'ametros \& inicializaci\'on de feromonas \;
%      \While{Criterio de finalización no se cumple}{
%       Construcci\'on\_de\_soluci\'on\_de\_cada\_hormiga()\;
%       M\'etodo\_de\_b\'usqueda\_no\_local(); //Paso opcional\\
%       Actualizaci\'on\_de\_feromonas()\;
%      }
%      \caption{Algoritmo de un modelo COP adaptado a una metaheur\'istica ACO}\label{COP-ACO-Algo}
%     \end{algorithm}
% \end{frame}


\section{Algoritmo Propuesto}
\subsection{Inicializaci\'on de Par\'ametros}
\begin{frame}{Inicializaci\'on de Par\'ametros en ACO}
    \begin{columns}
        \hspace{-1cm}
        \begin{column}{0.45\textwidth}
            \begin{itemize} \fontsize{9pt}{5}\selectfont
                \item Umbral $\theta$: 
                \begin{itemize} \fontsize{9pt}{5}\selectfont
                    \item $[0, \theta]$
                    \item $]\theta, Max\_Angle]$
                \end{itemize}
                \item Umbral $Max\_Angle$: 
                \begin{itemize} \fontsize{9pt}{5}\selectfont
                    \item $max(2.5 \theta, 90\degree)$
                    \item $\measuredangle (e_{i}, e_{j})> Max\_Angle$
                \end{itemize}
                \item Factor {\it Max\_Axial\_Displacement}: Curvatura y Magnitud entre segmentos
                \item {\it Max\_Score}
                \item Valor Inicial $\tau_0$
                \item Criterio de finalizaci\'on de un recorrido
            \end{itemize}
        \end{column}
        \begin{column}{0.55\textwidth}
        %\vspace{-2cm}
            \centering
            \includegraphics[scale=0.55]{Pictures/ant-params.png}
        \end{column}
    \end{columns}
\end{frame}

\subsection{Construcci\'on de Soluciones}
\begin{frame}{ACO: Construcci\'on de Soluci\'on}
\centering
\includegraphics[scale=0.35]{Pictures/ACO-ant-Constr.png}
\begin{itemize}
    \item Influencia de feromonas $\tau$ y de una heur\'istica $\eta$ en la elecci\'on de Aristas/Componentes $c_i$ que respeten restricciones en $\Omega$
    \item Aspecto aleatorio en la elecci\'on
\end{itemize}
\end{frame}

% \begin{frame}{ACO: Heur\'istica de Asignaci\'on Inicial}
% \begin{enumerate}
% \item Arista $a: (n_i,n_j)$ tal que $deg(n_i) = 1$ o $deg(n_j) = 1$ 
% %La arista a asignar debe tener al menos uno de sus nodos con grado 1, indicando que es el inicio o final de una parte del grafo.

% \item De no haber aristas disponibles con esas caracter\'isticas, se realiza una asignaci\'on inicial de una arista que cumpla con 2 criterios:
% \begin{itemize}
%     \item $a: (n_i,n_j)$ tal que $deg(n_i) >= 2$ o $deg(n_j) >= 2$ 
%     \item El \'angulo que forma la arista candidata a elegir, junto a otra arista a la que pertenece el nodo, debe pertenecer al rango $]\theta, Max\_Angle]$.
% \end{itemize}

% \item Una arista aleatoria que no pertenezca a una soluci\'on o camino evaluado como de buena calidad
% %. La calidad de un camino se presenta m\'as adelante en esta secci\'on.
% \end{enumerate}
% \end{frame}

\begin{frame}{ACO: Construcci\'on de Soluci\'on, 1ra Arista}
    \centering
    \includegraphics[scale=0.5]{Pictures/ant-initial-edge.png}
    \begin{itemize}
        \item Heur\'istica de Asignaci\'on Inicial: 3 criterios
        \item Enfoque en la exploraci\'on del espacio de b\'usqueda
    \end{itemize}
\end{frame}

\begin{frame}{ACO: Construcci\'on de Soluci\'on}
\centering
\includegraphics[scale=0.35]{Pictures/ACO-ant-Constr-choices.png}
        \begin{equation}
P(c_{i} | s^{P}) = \frac
        {\tau_{i}^{\alpha} ~ \eta_{i}^{\beta}}
        {\sum\limits_{c_{i}\in N(s^p)}{\tau_{i}^{\alpha} ~ \eta_{i}^{\beta} } }, \forall c_{i} \in N(s^{P}).
\label{eq:antProbabilities}
\end{equation}
\end{frame}

\begin{frame}{ACO: Construcci\'on de Soluci\'on}
\centering
\includegraphics[scale=0.35]{Pictures/ACO-ant-Constr-choices.png}
\begin{itemize}
    \item Calidad $s^{P}_1$ = $\sum \eta_{i}$
    \item Calidad $s_1$ = $\frac{1}{n}\sum \eta_{i} $
    \item Calidad M\'inima $\geq \frac{Max\_Score}{2}$
    \item Buena Calidad: $Max\_Score$
    \item Calidad Intermedia: $[\frac{Max\_Score}{2}, Max\_Score[$
\end{itemize}
\end{frame}

\subsection{M\'etodo de b\'usqueda no local}
\begin{frame}{B\'usqueda no local: l\'ogicas globales/centralizadas}
    
    \begin{columns}
        \begin{column}{0.4\textwidth}
            \begin{itemize}
                \item Eliminar soluciones candidatas que no aporten informaci\'on nueva
                \item Soluciones parcialmente repetidas
            \end{itemize}
        \end{column}
        \begin{column}{0.2\textwidth}
        \includegraphics[scale=0.5]{Pictures/ant-segments-repetead-sol1.png}
        \end{column}
        \begin{column}{0.2\textwidth}
        \includegraphics[scale=0.5]{Pictures/NoConsenso2.png}
        \end{column}
        \begin{column}{0.2\textwidth}
        \includegraphics[scale=0.5]{Pictures/NoConsenso3.png}
        \vspace{0.5cm}
        \includegraphics[scale=0.5]{Pictures/NoConsenso4.png}
        \end{column}
    \end{columns}
\end{frame}

\begin{frame}{ACO: Actualizaci\'on de Feromonas}
\centering
\includegraphics[scale=0.4]{Pictures/ACO-ant-ferom.png}
\begin{itemize}
    \item Evaluaci\'on de factiblidad de soluciones candidatas $s \in S$
    \item Premiar los caminos de buena calidad
    \item Prop\'osito: Convergencia sobre un camino \'optimo
\end{itemize}
\end{frame}


\subsection{Anti-feromonas}
\begin{frame}{Anti-feromonas SAP}
\centering
\includegraphics[scale=0.4]{Pictures/ACO-ant-ferom-penalize.png}
    \begin{itemize}
        \item Cambio: $\tau_0 \longrightarrow \gamma$
        \item $\gamma$ es un factor penaliza los caminos de mala calidad
        \item 2 penalizaciones $\longrightarrow \tau_i = 0$ 
    \end{itemize}
\end{frame}

\begin{frame}{Problema de Anti-feromonas SAP}
\centering
\includegraphics[scale=0.4]{Pictures/ACO-ant-constr-penalize.png}
    \begin{itemize}
        \item Problema: penalizaci\'on puede ocasionar perdida de soluciones factibles
        \item Propuesta: Relacionar penalizaci\'on de $\tau_i$ con $c_i \in s^P$
    \end{itemize}
\end{frame}

\begin{frame}{Anti-feromonas SAP dependientes del camino previo}
    \includegraphics[scale=0.5]{Pictures/ACO-ant-ferom-penalize-seg.png}
\end{frame}


\begin{frame}{Extensi\'on de Anti-feromonas sobre un segmento}
\centering
\includegraphics[scale=0.4]{Pictures/ant_segments_simple_case.png}
    \begin{itemize}
        \item Cada arista del segmento forma un \'angulo en el rango $[0, \theta]$ con sus vecinos
        \item Penalizaci\'on de $e_3$ se asocia al $seg^{b}_{1}$, sin perjudicar otros caminos que pasen por $e_3$ pero no por el segmento $seg^{b}_{1}$.
    \end{itemize}
\end{frame}

\begin{frame}{Anti-feromonas sobre un segmento de una sola arista}
\centering
\begin{figure*}
  \begin{subfigure}[t]{0.48\textwidth}
    \includegraphics[scale=0.5]{Pictures/ant_segments_complex_case_B2.png}
  \end{subfigure}%
    ~ \hspace{0.1cm}
    \begin{subfigure}[t]{0.48\textwidth}
    \includegraphics[scale=0.5]{Pictures/ant_segments_complex_case_B_blocked.png}
    \end{subfigure}
\end{figure*}
\begin{itemize}
    \item Segmentos de una sola arista ocasionan mismo problema que se intenta resolver
\end{itemize}
\end{frame}

\begin{frame}{Anti-feromonas sobre un segmento}
\centering
    \includegraphics[scale=0.5]{Pictures/ant_segments_complex_case_B2_extended.png}
    \begin{itemize}
        \item Se extiende el segmento unitario con el segmento que lo precede
    \end{itemize}
\end{frame}

\subsection{Criterios para la Actualizaci\'on de Anti-feromonas}
\begin{frame}{Anti-feromonas sobre soluciones de calidad intermedia}
\begin{itemize}
    \item Curvatura de una soluci\'on
    \item Magnitud de desplazamiento entre segmentos
    \item Criterio espec\'ifico para neuronas
\end{itemize}
\end{frame}

\begin{frame}{Curvatura de una soluci\'on}
\centering
    \includegraphics[scale=0.5]{Pictures/ant_curvature_case.png}
    \begin{itemize}
        \item El \'angulo entre la proyecci\'on de $\Vec{p}$ y $\Vec{q}$ debe ser menor a $\theta \times Max\_Axial\_Displacement$ para no descartar la soluci\'on.
    \end{itemize}
\end{frame}

\begin{frame}{Magnitud de desplazamiento entre segmentos}
    \begin{figure*}[h]
    \centering
    \begin{subfigure}[t]{0.45\textwidth}
        \centering
        \includegraphics[scale=0.4]{Pictures/ant_segmentMagnitude_case.png}
    \end{subfigure}
    ~ \hspace{0.5cm}
    \begin{subfigure}[t]{0.45\textwidth}
        \centering
        \includegraphics[scale=0.4]{Pictures/ant_segmentMagnitude_case_2.png}
    \end{subfigure}%
    \begin{itemize}
        %\item El criterio de curvatura puede no ser suficiente por si mismo
        \item Comportamiento esperado de la rigidez permite descartar soluciones con cambios demasiado pronunciados entre sus segmentos
    \end{itemize}
\end{figure*}
\end{frame}

%\subsection{Extracci\'on de informaci\'on para individualizar filamentos}
\begin{frame}{Criterio espec\'ifico para neuronas}
    \begin{itemize}
        \item Se debe validar que los filamentos de una neurona parten del {\bf soma}
        \item  Extracci\'on de informaci\'on: caracter\'isticas geom\'etricas, topol\'ogicas y espaciales
    \end{itemize}
    \begin{columns}
        \begin{column}{0.55\textwidth}
        \centering
        \includegraphics[height=1.7in]{Pictures/Porta183-somaEdges-example2.png}
        \end{column}
        \begin{column}{0.45\textwidth}
            \begin{equation}
                \tau_{ij} \leftarrow
                    \begin{cases}
                     \tau_{ij} \cdot \gamma \text{ si } deg(v^{a}_{n}) = 1,  \\[3ex]
                    
                    \text{0 si } \tau_{ij} \leq 0.25, \\[3ex]
                    \tau_{ij} \quad \text{en otro caso}.
                    \end{cases}
            \end{equation}
        \end{column}
    \end{columns}
    
\end{frame}

% \begin{frame}{Extracci\'on de informaci\'on para individualizar filamentos}
%      \begin{figure*}[h!]
%     \centering
%     \begin{subfigure}[t]{0.48\textwidth}
%         \centering
%         %\includegraphics[height=1.5in]{Pictures/50-ROIs-Spinning-Marchantia-somaEdges.png}
%     \end{subfigure}%
%     ~ \hspace{0.1cm}
%     \begin{subfigure}[t]{0.48\textwidth}
%     \centering
%         \includegraphics[height=1.5in]{Pictures/Porta18-3a1-somaEdges.png}
%     \end{subfigure}
%     \end{figure*}
%     \begin{itemize}
%         \item Caracter\'isticas geom\'etricas, topol\'ogicas y espaciales
%     \end{itemize}
% \end{frame}