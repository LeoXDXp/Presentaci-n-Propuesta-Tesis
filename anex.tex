\addtocontents{toc}{\protect\setcounter{tocdepth}{0}}
\section{Anexo}

\begin{frame}{Heur\'istica Miope}
    \begin{equation}%\fontsize{9pt}{7.2}\selectfont
    \eta_{ij} = 
        \begin{cases} 
        \text{Max\_Score si } \measuredangle(c_{ij}, s_{n}^{P}) \in [0, \theta],\\[3ex]
        
        \text{Max\_Score} \cdot \left(1 - \dfrac{ \left| \measuredangle(c_{ij}, s_{n}^{P}) - \frac{\theta}{2} \right|} {180} \right)  \text{ si } \measuredangle(c_{ij}, s_{n}^{P}) \in \quad ]\theta, \text{Max\_Angle}],\\[3ex]
        
        \text{0 en otro caso.}
        \end{cases}
    \end{equation}    
\end{frame}

\begin{frame}{VI, \'Indice Rand e \'Indice Jaccard}
\resizebox{\textwidth}{!}{
    \begin{tabular}{|c|c|c|c|c|c|}
        \hline
        Figura & Config. de Par\'ametros & VI Max & VI & Rand & Jaccard \\ \hline
         4.1 & MT-P  & 2.3978 & 1.6360 & 0.7646 & 0.2485 \\
         4.1 & S1  & 2.3978 & 0.7091 & 0.8637 & 0.4750  \\
         4.2 & MT-P & 3.0445 & 2.2296 & 0.7276 & 0.1806 \\
         4.2 & S2 & 3.0445 & 2.5878 & 0.7276 & 0.1673  \\
         4.3 & MT-P & 3.4965 & 2.1656 & 0.8658 & 0.2407 \\
         4.4 & MT-P & 3.7612 & 2.6285 & 0.8683 & 0.2488  \\
         4.5 & MT-P & 1.9459 & 0.4286 & 0.8929 & 0.40 \\
         4.6 & N & 6.0258 & 1.7950 & 0.8864 & 0.1389 \\
         4.7 & N & 5.0814 & 3.7256 & 0.8775 & 0.0703 \\
         4.8 & N & 4.2046 & 1.0060 & 0.8794 & 0.2157 \\ \hline
    \end{tabular}
    %\caption{VI $\in [0, VI\_Max]$, Rand y Jaccard $\in [0, 1]$ }
}
\end{frame}

\begin{frame}{VI, \'Indice Rand e \'Indice Jaccard}
%    La mayor\'ia de los criterios para comparar particiones mediante conteo de pares   se basa en la 
    
    %suele fundamentarse en el uso de la matriz de confusi\'on, tambi\'en llamada matriz de asociaci\'on o tabla de contingencia \cite{meilua2007comparing}. Esta tabla considera 4 casos en los que puede estar un par de elementos del {\it data set}, que para el caso de la individualizaci\'on de filamentos son pares de aristas, en las particiones $C$ y $C'$:

    \begin{itemize}\fontsize{9pt}{10}\selectfont
        \item Mayor\'ia de los criterios para comparar particiones $\rightarrow$ matriz de confusi\'on/asociaci\'on o tabla de contingencia.
        \item $N_{11}$: N\'umero de pares que est\'an en el mismo cluster en $C$ y $C'$
        \item $N_{00}$: N\'umero de pares que est\'an en distintos clusters en $C$ y $C'$
        \item $N_{10}$: N\'umero de pares que est\'an en el mismo cluster en $C$ pero no en $C'$
        \item $N_{01}$: N\'umero de pares que est\'an en el mismo cluster en $C'$ pero no en $C$
        \item $N_{11} + N_{00} + N_{10} + N_{01} = \frac{n(n-1)}{2}$, con n como el n\'umero de aristas.
        \item Asociaci\'on de casos con evaluaciones de clasificaci\'on
    \end{itemize}

    %\resizebox{\textwidth}{!}{
    \makebox[\linewidth][c]{\fontsize{10pt}{12}\selectfont
        \begin{tabular}{|c|c|}
        \hline
            Casos para un par de aristas & Clasificaci\'on \\ \hline
            $N_{11}$ & Verdadero Positivo ({\it True Positive} o TP) \\
            $N_{00}$ & Verdadero Negativo ({\it True Negative} o TN)\\
            $N_{10}$ & Falso Positivo ({\it False Positive} o FP)\\
            $N_{01}$ & Falso Negativo ({\it False Negative} o FN)\\ \hline
        \end{tabular}
    }
\end{frame}

\begin{frame}{\'Indice Rand y Jaccard}
    \begin{equation}\fontsize{9pt}{7.2}\selectfont
        \mathcal{R}(C,C^{\prime}) = \frac{N_{11} + N_{00}}{n(n-1)/2} = \frac{TP + TN}{TP + TN + FP + FN}
    \end{equation}
    \vspace{0.2cm}
    \begin{equation}\fontsize{9pt}{7.2}\selectfont
        \mathcal{J}(C,C^{\prime}) = \frac{N_{11}}{N_{11} + N_{01} + N_{10}} = \frac{TP}{TP + FP + FN}
    \end{equation}
    \vspace{0.2cm}
    \begin{equation}\fontsize{9pt}{7.2}\selectfont
        \text{Precision} = \frac{TP}{TP + FP}
    \end{equation}
    \vspace{0.2cm}
    \begin{equation}\fontsize{9pt}{7.2}\selectfont
        \text{Recall} = \frac{TP}{TP + FN}
    \end{equation}
\end{frame}


\begin{frame}{Curvatura de una soluci\'on y Magnitud de desplazamiento entre segmentos}
    \centering
    \begin{equation}\fontsize{9pt}{7.2}\selectfont
    \tau_{ij} \leftarrow
        \begin{cases}
        \tau_{ij} \cdot \gamma \text{ si } \measuredangle(proy(\Vec{p}), \Vec{q}) \geq \theta \cdot \text{Max\_Axial\_Displacement},\\[3ex]
        
        \text{0 si } \tau_{ij} \leq 0.25, \\[3ex]
        \tau_{ij} \quad \text{en otro caso.}
        \end{cases}
    \end{equation}
    \vspace{1cm}
    \begin{equation}\fontsize{9pt}{7.2}\selectfont
    \tau_{ij} \leftarrow
        \begin{cases}
        \begin{split}
         \tau_{ij} \cdot \gamma \text{ si } & \sin(\measuredangle seg^{a}_{pMaxMin}) > seg^{a}_{iMax} \cdot 0.1 \cdot \text{Max\_Axial\_Displacement} \\ & \land \measuredangle seg^{a}_{pMaxMin}) > \theta,
        \end{split}
        \\[3ex]
        
        \text{0 si } \tau_{ij} \leq 0.25, \\[3ex]
        \tau_{ij} \quad \text{en otro caso}.
        \end{cases}
\end{equation}
\end{frame}

% \begin{frame}{Ponderaci\'on de las caracter\'isticas}
%     \begin{itemize}\fontsize{9pt}{12}\selectfont
%     \item No es posible utilizar un \'unico par\'ametro para obtener una ponderaci\'on directa
%     %El uso de diversas caracter\'isticas en distintas partes del algoritmo propuesto
    
%     %Dado que las diversas caracter\'isticas utilizadas en el algoritmo propuesto se encuentran en distintos m\'etodos de la metaheur\'istica ACO, 
%     \item Construcci\'on de soluciones: $\frac{100}{n}$\% el grado de los nodos, y el remanente para el \'angulo entre aristas, con $n$ como el n\'umero de aristas.
%     \item M\'etodo de b\'usqueda no local: 100\% posici\'on de la arista.
%     \item Actualizaci\'on de feromonas: 50\% curvatura, 50\% diferencia de magnitud entre segmentos. En el caso espec\'ifico de las neuronas, corresponde a $33.\overline{3}\%$ curvatura, $33.\overline{3}\%$ diferencia de magnitud entre segmentos y $33.\overline{3}\%$ informaci\'on topol\'ogica de centralidad.
% \end{itemize}
% \end{frame}