\documentclass[aspectratio=169]{beamer}
\usetheme{KUL}
\usepackage{multirow}
\usepackage{multicol}
\usepackage{tikz}
\usepackage{ulem}
\usepackage{siunitx}
\newcommand\itemS{\item[\textbf{\S}]}
\definecolor{darkgreen}{rgb}{0,0.598,0.199}
\usepackage{times} % set font on times new roman
\usepackage{eurosym} % package for Euro sign
\usepackage{lineno}   % package for line numbering
\usepackage{hyperref} % this is for url links
\usepackage{subcaption}  % this package enables one to put several figures next to each other
%\usepackage{capt-of}
\usepackage{textcomp}
\usepackage{setspace}
\usepackage{gensymb}
%\usepackage[font=small]{caption}
\usepackage{diagbox}
\usepackage[ruled,vlined,spanish,onelanguage]{algorithm2e}
\newenvironment{figure*}%
{\begin{figure}}
{\end{figure}}
\captionsetup[figure]{labelformat=empty, font=scriptsize}
\captionsetup[table]{labelformat=empty, font=scriptsize}
%\captionsetup{font=scriptsize, labelfont=scriptsize}
\usepackage{pgfpages}
\setbeameroption{show notes on second screen=right} % Both
\setbeamertemplate{note page}[plain]

%----------------------------------
% Fill in the essential Information
%----------------------------------

\title[Individualizaci\'on de filamentos mediante optimizaci\'on]{Charla de Tesis II: Individualizaci\'on de filamentos en una red mediante optimizaci\'on}
%\subtitle{\ldots a subtitle}
\author[L.\ Pizarro]{Leonardo Pizarro} % between [] is short name, between {} is long name
\date{\today} % Here you can also just type something, e.g. October 10, 2017
%\institute[DCC - FCFM - UChile]{Magister en Ciencias de la Computaci\'on\\Facultad de Ciencias F\'isicas y Matem\'aticas\\ Departamento de Ciencias de la Computaci\'on \\ Universidad de Chile}
\institute[]{Profesor Gu\'ia: Mauricio Cerda\\Profesor Co-gu\'ia: Jacques Dumais}

%----------------------------------
% ACTUAL PRESENTATION STARTS HERE
%----------------------------------

\begin{document}

% TITLE PAGE	
	{
		\setbeamertemplate{headline}{} %define local, empty header for title page
		\setbeamertemplate{footline}{} %define local, empty footer for title page
		\maketitle
	}
	\addtocounter{framenumber}{-1} % We don't count the title page

\begin{frame}
\frametitle{Contenidos} 
\tableofcontents
\end{frame}

\section{Motivaci\'on y Antecedentes}
% [allowframebreaks] para dividir en multiples frames
\begin{frame}{¿Qu\'e es un filamento?}
    \begin{columns}
    \begin{column}{0.4\textwidth}
        \begin{itemize}
            \item Estructuras alargadas con diferentes propiedades
            \item Observaci\'on de Microscop\'ia
            \begin{itemize}
            \small
                \item Ruido
                \item Resoluci\'on
            \end{itemize}
        \end{itemize}
    \end{column}
    \begin{column}{0.6\textwidth}
        %\begin{figure*}[h]
        \centering
            %\begin{subfigure}[t]{0.48\textwidth}
        \includegraphics[height=1.3in]{Pictures/small_MT.jpg}
            %\end{subfigure}
        ~ 
            %\begin{subfigure}[t]{0.48\textwidth}
        \includegraphics[height=1.3in]{Pictures/sun_filament.jpg}
            %\end{subfigure}
        \vspace{0.2cm}
             %\begin{subfigure}[t]{\textwidth}
        \includegraphics[height=1.3in]{Pictures/citoesqueletoo-min.png}
            %\end{subfigure}
        %\end{figure*}
    \end{column}
\end{columns}
\end{frame}

\note[itemize]{
\item En la naturaleza, es posible encontrar de forma ubicua, estructuras alargadas, que denominamos filamentos, que presentan distintas caracter\'isticas y comportamientos
\item Las caracter\'isticas y/o comportamientos de un filamento, o un a red de estos, puede develar informaci\'on relevante respecto al ambiente o al interior del elemento donde se encuentra el filamento

\item pudiendo ser esto una c\'elula o una estrella, 

\item Luego, una de las formas de recolectar esta informaci\'on de forma cuantitativa es mediante la individualizaci\'on de filamentos

\item enfoque de investigaci\'on en microscop\'ia

\item mt, actina e interm con tama\~nos inferiores a la resoluci\'on m\'axima de microscopios actuales


}

\begin{frame}{Varias soluciones posibles}
  \begin{columns}
    \begin{column}{0.6\textwidth}
        \centering
        \includegraphics[scale=0.5]{Pictures/NoConsenso.png}
    \end{column}
    \begin{column}{0.2\textwidth}
        \includegraphics[scale=0.5]{Pictures/NoConsenso2.png}
    \end{column}
    \begin{column}{0.2\textwidth}
        \includegraphics[scale=0.5]{Pictures/NoConsenso3.png}
        \vspace{0.5cm}
        \includegraphics[scale=0.5]{Pictures/NoConsenso4.png}
    \end{column}
    \begin{column}{0.2\textwidth}
    \end{column}
\end{columns}
\end{frame}

\note[itemize]{
    \item Derivado del problema de la resoluci\'on
    
    \item comparar individualizar con otros problemas que detectan e identifican, 
    la detección por si sola es compleja
    %Así, si se desea comparar el problema de individualizar filamentos con otros 
    %problemas similares, donde se debe detectar e identificar, en este caso, se tiene el
    \item Luego, a modo de resumen (pasar a sgte ppt)
}

\begin{frame}{Dificultades del Problema}
    \begin{enumerate}
        %\item Variaci\'on de las caracter\'isticas de filamentos entre diferentes c\'elulcas
        \item Desconocimiento de los puntos de inicio y fin de un filamento
        \item Necesidad de incorporar m\'as descriptores de los filamentos
        \item Resoluci\'on y Ruido en las im\'agenes
        \item Diferencias en el comportamiento de los filamentos dependiendo de la c\'elula observada.
    \end{enumerate}
\end{frame}

\note[itemize]{
    \item podemos indicar que estos 4 son los elementos que definen la base de 
    la individualizaci\'on
    
    \item el 1er punto dificulta bastante el utilizar herramientas o m\'etodos conocidos para recorrer caminos
    
    \item Estos puntos son abordados por separado en diferentes formas en algunos m\'etodos del estado del arte que veremos a continuaci\'on
}

\begin{frame}{Familias de M\'etodos}
    %\hspace{-1cm}
    \centering
    \includegraphics[height=2.5in]{Pictures/familiasDeMetodos.png}
\end{frame}

\note[itemize]{
    \item Clasificar en 2 grandes familias, donde ambas parten de una imagen, y realizan un procesamiento inicial de esta, pero se diferencian en si se basan s\'olo en el uso de herramientas de visi\'on por computador o si tambi\'en incorporar otras, como es el caso de los m\'etodos basados en optimizaci\'on
    \item Los m\'etodos basados principalmente en herramientas de visi\'on por computador tienden a realizar un an\'alisis local a nivel de p\'ixel, recopilando informaci\'on a partir del vecindario del p\'ixel.
    \item diferencia principal se observa en la cantidad o el tipo de caracter\'isticas, las que van aumentando a medida que se aborda un  problema de forma m\'as general. (hacer gesto de flecha)
    }

%\subsection{M\'etodos basados en Vision por Computador}
% \begin{frame}{Extracci\'on de redes de MT}
    
%     \begin{columns}
%     %\hspace{-1cm}
%         \begin{column}{0.25\textwidth}
%             \centering
%             \includegraphics[height=1.6in]{Pictures/MT-LFT.jpeg}
%         \end{column}
%         \begin{column}{0.75\textwidth}
%             \centering
%             \includegraphics[height=1.6in]{Pictures/MT-OFT.jpeg}
%         \end{column}
%     \end{columns}
%     %\vspace{1cm}
%     \begin{figure}
%         %\includegraphics{}
%         \caption{Zhen Zhang, Yukako Nishimura, and Pakorn Kanchanawong. Extracting microtubule networks from superresolution single-molecule localization microscopy data. Molecular biology of the cell, 28(2):333–345, 2017.}
%     \end{figure}
%     % \begin{itemize}\fontsize{8pt}{7.2}\selectfont
%     %     \item 
%     % \end{itemize}
% \end{frame}

% \note[itemize]{
%     \item discontinuidad de filamentos en una imagen, atribuyéndose a ruido o la densidad del componente fluorescente
%     \item Espec\'ifico para MTs
%     \item 2 filtros, LFT (FIltro de transf. lineal) y el OFT, Filt Transf Orientaci\'on
%     \item 
%     \item Criterio de similaridad de la direcci\'on proyectada , as\'i un 2do criterio asociado a la distancia entre los extremos finales de los segmentos 

% }

\begin{frame}{Quantitative IFS}
    \begin{columns}
        \begin{column}{0.6\textwidth}
             \begin{figure}
                \centering      \includegraphics[height=2.15in]{Pictures/QuantitativeIFS.png}
                \caption{Jun Qiu and F LI. Quantitative morphological analysis of curvilinear network for microscopic image based on individual fibre segmentation (ifs). Journal of microscopy, 256(3):153–165, 2014}
            \end{figure}       
        \end{column}
        \begin{column}{0.4\textwidth}
            \begin{itemize}\fontsize{9pt}{7.2}\selectfont
                \item 6 filtros sobre la imagen
                \item {\it Short Branch Deleted Algorithm} realiza una clasificación de intersecciones
                %\vspace{-0.5cm}
                \begin{itemize}\fontsize{9pt}{7.2}\selectfont
                    \item Nodos terminales
                    \item Nodos Aislados
                    \item Nodos puentes
                    \item Intersecciones
                \end{itemize}
                \item Resultado: Fibras Aisladas (verde), superpuestas (púrpura) y otras (azul, rojo)
            \end{itemize}
        \end{column}
    \end{columns}
\end{frame}

\note[itemize]{
    \item El objetivo de esta investigaci\'on es realizar un estudio topol\'ogico a nivel de p\'ixeles de acuerdo a estructuras predefinidas
    \item Se enfoca en clasificar las intersecciones para diferencias si los segmentos de filamento se encuentran superpuestos (p\'urpura), aislados(verde) y otros (azul, rojo)

    \item Se utilizan 6 filtros para procesar una imagen, resaltandose el filtro de esqueletonizaci\'on y el de estructuras alargadas
    
    \item El an\'alisis de estructuras predefinidas se denomina SBDA
}

%\subsection{M\'etodos basados en Optimizaci\'on}
\begin{frame}{SOAX: Curvas param\'etricas (SOAC) para identificar filamentos}
    \begin{columns}
        \begin{column}{0.65\textwidth}
             \begin{figure}
                \centering
                \includegraphics[height=2in]{Pictures/SOAX_translated.jpg}
                \caption{Ting Xu, Dimitrios Vavylonis, Feng-Ching Tsai, Gijsje H Koenderink, Wei Nie, Eddy Yusuf, I-Ju Lee, Jian-Qiu Wu, and Xiaolei Huang. Soax: a software for quantification of 3d biopolymer networks. Scientific reports, 5:9081, 2015.}
            \end{figure}       
        \end{column}
        \begin{column}{0.35\textwidth}
            \begin{itemize}\fontsize{9pt}{10.2}\selectfont
                \item 1 SOAC = 1 Segmento de Filamento
                \item 4 Par\'ametros: $\tau$, $K_{str}$, $t$, $c$
                \item Funci\'on Objetivo $F = -L_{total} + cL_{<t}$
            \end{itemize}
        \end{column}
    \end{columns}
\end{frame}

\note[itemize]{
    \item curvas parametricas de contorno abierto, que se elongan o estiran
    %\item SOAC es la sigla en ingles para curvas elongadas de contorno abierto
    \item Estas curvas nacen de puntos de alta intensidad, a partir de los cuales crecen/estiran hasta cruzarse con otra
    \item  Asi, se identifican las uniones que representan nodos en una red de filamentos
    \item Algunos de estos par\'ametros definen el crecimiento/elongamiento de las curvas, mientras que el factor c influye directamente en la funci\'on objetivo
    \item se busca minimizar F, ajustando los par\'ametros, y obteniendo una red en la que no se puede explorar individualmente los filamentos
}

\begin{frame}{DeFiNe}
    %\vspace{-1cm}
    \begin{columns}
        \begin{column}{0.55\textwidth}
            \begin{itemize}\fontsize{9pt}{5}\selectfont
                %\item Conjunto de aristas $\rightarrow$ camino/filamentos
                \item Problema parte de un grafo
                \item Recorrer los P caminos es NP-Hard
                %\item Similitud con un {\it Path Cover Problem} $\rightarrow$ FCP
                \item 2 Heur\'isticas para crear {\bf P'}
            \end{itemize}        
        \end{column}
        \begin{column}{0.45\textwidth}
            \begin{itemize}\fontsize{9pt}{7.2}\selectfont
                
                \item {\bf P'} $\rightarrow$ {\it Set Cover Problem}
                \item {\it SCP} usa 1 propiedad para ponderar aristas
            \end{itemize}
        \end{column}
    \end{columns}
    \begin{figure}
        \centering
        \includegraphics[scale=0.42]{Pictures/flujoDefine2.png}
        \caption{David Breuer and Zoran Nikoloski. Define: an optimisation-based method for robust disentangling of filamentous networks. Scientific reports, 5:18267, 2015.}
    \end{figure}
\end{frame}

\note[itemize]{
    \item Los autores definen que su problem\'atica parte desde un grafo ponderado que representa una red de filamentos, y que la extracci\'on de esta red desde una imagen es un problema previo.
    \item Asi, un filamento se representa como un conjunto de aristas adyacentes, cuyas combinaciones posibles se definen como P. Recorrer estos P filamentos para evaluarlos es NP-Hard debido al n\'umero de combinaciones posibles entre las aristas
    \item El problema de optimizaci\'on que DeFiNe resuelve se denomina FCP, y dado que es NP-Hard, los autores definen 2 heur\'isticas para extraer un subconjunto de filamentos posibles desde P, creando P prima
    \item El uso de alguna de las heur\'isticas definidas para DeFiNe permite que P prima sea resuelto en tiempo polinomial, mediante la resoluci\'on de un Set Cover Problem, donde P prima es el dato de entrada
    \item Como resultado, se pueden individualizar los filamentos
    \item Los 2 problemas de este enfoque son que utiliza solo 1 propiedad asociada a las aristas para individualizar filamentos, y que DeFiNe no escale de buena manera cuando el grafo crece
}


\begin{frame}{Problemas Principales de la Individualizaci\'on de Filamentos}
    \begin{itemize}
    \item M\'etodos presentados presentan problemas asociados
    \item Se desconoce a priori el n\'umero de filamentos a buscar.%, dado que una imagen puede tener individualizaciones distintas para 2 expertos.
    \item Se busca individualizar m\'as de un filamento por imagen
    \item Las combinaciones de aristas en un grafo crecen de manera exponencial
    
\end{itemize}
\end{frame}


\note[itemize]{

    \item Los m\'etodos presentados, tanto los que individualizan como los que no lo hacen, presentan problemas asociados a la herramienta primaria que utilizan. 
    \item Los que se basan principalmente {\bf s\'olamente} en herramientas de visi\'on por computador tienen un alto n\'umero de par\'ametros, mientras que los basados en optimizaci\'on pueden tener un alto costo computacional

    \item A esto, le agregamos los problemas espec\'ificos, como son
    
    \item Se desconoce a priori el n\'umero de filamentos a buscar dado que una imagen puede tener individualizaciones distintas para 2 expertos.
    \item Generalmente se busca individualizar m\'as de un filamento por imagen, lo que conlleva a elegir los mejores filamentos entre las soluciones que se encuentren.
    \item El uso de un grafo para representar la red de filamentos puede implicar que las combinaciones de soluciones crezcan de manera exponencial.
}

\section{Hip\'otesis y Objetivos}
\begin{frame}{Hip\'otesis}
\small
    \begin{itemize}
        \item {\bf A partir de un grafo con pesos, no dirigido, con o sin ciclos, que representa una red de filamentos}, en conjunto con utilizar una combinaci\'on de caracter\'isticas de los segmentos de filamentos como largo, grosor, \'angulo de bifurcaci\'on o curvatura, sumado a la incorporaci\'on de informaci\'on previa disponible sobre el tipo de c\'elula (red de filamentos con o sin ciclos), {\bf es posible identificar a cada filamento en la red resolviendo un modelo de optimizaci\'on}.
    \end{itemize}
\end{frame}

\begin{frame}{Objetivo General}
\small
    \begin{itemize}
        \item Desarrollar un modelo de optimizaci\'on para la individualizaci\'on de filamentos a partir de un grafo representativo de la red de filamentos, evaluando la variaci\'on en el resultado con distintas combinaciones entre las propiedades de cada segmento de filamento como el grosor, largo, \'angulo de bifurcaci\'on o direcci\'on.
    \end{itemize}
\end{frame}


\begin{frame}{Objetivos Espec\'ificos}
\small
\begin{enumerate}
    \item[OE 1] Generar un modelo de optimizaci\'on %para la identificaci\'on de filamentos a partir de un grafo con pesos que representa la red de filamentos.
    \item[OE 2] Implementar un algoritmo que resuelva el modelo de optimizaci\'on %, entregando como salida la identificaci\'on de filamentos, considerando casos de solapamiento y/o cruce.
    \item[OE 3] Identificar la ponderaci\'on de propiedades en casos espec\'ificos: microt\'ubulo y neurona %que entregue mejores resultados % para grafos que representen una neurona, una bacteria y una c\'elula eucariota de planta.
    \item[OE 4] Evaluar y comparar t\'ecnicas del estado del arte% que usan s\'olo herramientas de visi\'on por computador, basada en poblaci\'on de p\'ixeles, o que utilizan un m\'etodo derivado de contornos activos.
    %una soluci\'on exacta o aproximada respecto a la individualizaci\'on, dependiendo de la complejidad computacional del problema. 
\end{enumerate}
\end{frame}

\note[itemize]{

\item Generar un modelo de optimizaci\'on para la identificaci\'on de filamentos a partir de un grafo con pesos que representa la red de filamentos.
    \item Implementar un algoritmo que resuelva el modelo de optimizaci\'on, entregando como salida la identificaci\'on de filamentos, considerando casos de solapamiento y/o cruce.
    \item Identificar la ponderaci\'on de propiedades que entregue mejores resultados para grafos que representen una neurona, una bacteria y una c\'elula eucariota de planta.
    \item Evaluar t\'ecnicas que usan s\'olo herramientas de visi\'on por computador \cite{boudaoud2014fibriltool}, basada en poblaci\'on de p\'ixeles, o que utilizan un m\'etodo derivado de contornos activos \cite{xu2015soax}.

}

\section{Hormigas}
%\subsection{Metaheur\'istica ACO }
\begin{frame}{Hormigas (OE 1)}
    \centering
    \includegraphics[height=2.5in]{Pictures/ACO-ant.png}
\end{frame}

\note[itemize]{
    \item Como se mencion\'o previamente, uno de los problemas principales al individualizar filamentos es que se desconoce el comienzo y el final de un filamento al comienzo del problema. 
    \item As\'i, resulta natural explorar el comportamiento de las hormigas al buscar alimento como inspiraci\'on para individualizar filamentos. Este comportamiento se puede resumir en las 3 etapas que se muestran en la imagen.
    \item En una primera etapa, a la izquierda de la imagen en la letra {\bf a}, se puede observar que una hormiga que sale del hormiguero recorre un camino aleatorio hasta llegar al alimento. En su retorno, por el mismo camino, deposita feromonas para indicarle a otras hormigas por donde fue su recorrido.
    
    \item Luego, otras hormigas pueden tomar esta informaci\'on en su propio descubrimiento de un camino hacia el alimento, como se observa el la mitad de la imagen, en {\bf b}
    
    \item Finalmente, a la derecha de la imagen, se observa que las hormigas depositan m\'as y m\'as feromonas en el camino m\'as corto entre el hormiguero y el alimento, causando una convergencia sobre un camino \'optimo. Esta convergencia es lo que se define como el ciclo de reforzamiento positivo

    \item La comunicaci\'on indirecta entre las hormigas que conduce a la convergencia sobre un camino \'optimo se describe como el modelo de feromonas
}



\begin{frame}{Modelo de Feromonas para la Individualizaci\'on de Filamentos (OE 1)}
\small
\begin{itemize}
    \item Comunicaci\'on indirecta mediante feromonas
    \item B\'usqueda de una o m\'as soluciones
    \item Recorrido aleatorio
    \item Representaci\'on de filamentos mediante un grafo
\end{itemize}
    \begin{algorithm}[H]
    \SetAlgoLined
     Ajuste de Par\'ametros \& inicializaci\'on de feromonas\;
     \While{Criterio de finalizaci\'on no se cumple}{
       Construcci\'on\_de\_soluci\'on\_de\_cada\_hormiga()\;
       M\'etodo\_de\_b\'usqueda\_no\_local(); //Paso opcional\\
       Actualizaci\'on\_de\_feromonas()\;
     }
     \caption{Algoritmo metaheur\'istica ACO}\label{ACO-Algo}
    \end{algorithm}
\end{frame}

\note[itemize]{
    \item Este modelo sirve de inspiraci\'on para la metaheur\'istica ACO, mediante la cual es posible encontrar no solo una \'unica soluci\'on, sino que tambi\'en un conjunto de soluciones.
    
    \item  La representaci\'on de un conjunto de caminos en ACO puede ser interpretado como un grafo, donde cada camino corresponde a un filamento. 
    \item En si, la metaheur\'istica ACO cuenta con 4 etapas, 3 de las cuales se encuentran en un ciclo y cuyo puede variar con respecto a lo que se presenta en el algoritmo 1. Esta flexibilidad es una de las caracter\'isticas de la metaheur\'istica.
}

\begin{frame}{Etapas del Algoritmo Propuesto (OE 1)}
\centering  
\includegraphics[height=2.8in]{Pictures/ACOdiagram.png}
\end{frame}

\note[itemize]{
    \item En particular,  el uso de ACO para la individualizaci\'on de filamentps se refleja en el diagrama, siendo ACO la parte central del algoritmo propuestp.
    \item Para preparar los datos de entrada para ACO, es necesario extraer desde una imagen un grafo que represente una red de filamentos, utilizando alguna herramienta externa.
    \item Adicionalmente es posible alimentar algunos par\'ametros de entrada de ACO en el algoritmo propuesto mediante informaci\'on conocida a priori del comportamiento de la c\'elula observada en la imagen de la que se extrae el grafo.
    \item Con lo anterior, el algoritmo propuesto entra al ciclo donde se encuentran las 3 etapas descrita en el algoritmo 1: La construcci\'on de caminos, la actualizaci\'on de feromonas y el m\'etodo de b\'usqueda no local
}

%\subsection{Problema de Satisfacci\'on de Restricciones}
% \begin{frame}{ACO aplicado como Problema de Satisfacci\'on de Restricciones (OE 1)}
%   \begin{columns}
%     \begin{column}{0.5\textwidth}
%         \begin{itemize}
%           \item $P = (S, \Omega, F)$
%           % esta definido por un conjunto discreto de variables
%           \item S:  Cjto. de soluciones, compuesto por variables discretas $X_{i}$, $i \in 1 \dotsc n$
%           %\item $S:\quad X = v_{i}^{j} \in D_{i} = \{v_{i}^{1} \dotsc  v_{i}^{|D_{i}|}\}$.
%           \item Sol. Candidata $s \in S$, y $s^{*}$ una soluci\'on \'optima
%           \item Cjto. de Restricciones $\Omega$
%           \item Funci\'on objetivo $F: S\rightarrow \mathbb R_{0}^{+}$
%           %\item  
          
%       \end{itemize}
%     \end{column}
%     \begin{column}{0.5\textwidth}
%       \begin{itemize}
%           \item $s$ es factible si satisface las restricciones de $\Omega$
%           % y se relacionan mediante 
%           \item Pueden existir m\'ultiples soluciones $s^{*}$
%           \item $S^{*}$ es el conjunto que engloba todas las soluciones \'optimas
%           \item $s^{*} \in S^{*} \subseteq S$
%       \end{itemize}
%     \end{column}
% \end{columns}
% \end{frame}





% \begin{frame}{Adaptaci\'on a Individualizaci\'on de Filamentos}
%     \begin{algorithm}[H]
%     \SetAlgoLined
%     \KwData{Variables $X_i \dotsc X_n$, dominios $D_1 \dotsc D_n$, Restricciones $\in \Omega$}
%     \KwResult{conjunto s\textquotesingle $ \subseteq S$ != $\emptyset$, si existen soluciones factibles}
%      Ajuste de Par\'ametros \& inicializaci\'on de feromonas \;
%      \While{Criterio de finalización no se cumple}{
%       Construcci\'on\_de\_soluci\'on\_de\_cada\_hormiga()\;
%       M\'etodo\_de\_b\'usqueda\_no\_local(); //Paso opcional\\
%       Actualizaci\'on\_de\_feromonas()\;
%      }
%      \caption{Algoritmo de un modelo COP adaptado a una metaheur\'istica ACO}\label{COP-ACO-Algo}
%     \end{algorithm}
% \end{frame}


\section{Algoritmo Propuesto}

\begin{frame}{ACO: Construcci\'on de Soluci\'on, Asignaci\'on de 1ra Arista (OE 1)}
    \begin{columns}
        \begin{column}{0.3\textwidth}
            \begin{itemize}
                \item Heur\'istica de Asignaci\'on Inicial: \begin{enumerate}
                    \item Arista con un nodo grado 1
                    \item Aristas en intersecciones
                    \item Arista aleatoria
                \end{enumerate}
                
            \end{itemize}
        \end{column}
        \begin{column}{0.7\textwidth}
            \centering
            \includegraphics[scale=0.5]{Pictures/ant-initial-edge.png}
        \end{column}
    \end{columns}
\end{frame}

%\subsection{Construcci\'on de Soluciones}
% \begin{frame}{ACO: Construcci\'on de Soluci\'on (OE 1)}
% \centering
% \includegraphics[scale=0.4]{Pictures/ACO-ant-Constr.png}
% \begin{itemize}
%     \item Influencia de feromonas $\tau$ y de una heur\'istica $\eta$ en la elecci\'on de Aristas/Componentes $c_i$ que respeten restricciones en $\Omega$
%     \item Aspecto aleatorio en la elecci\'on
% \end{itemize}
% \end{frame}

% \begin{frame}{ACO: Heur\'istica de Asignaci\'on Inicial}
% \begin{enumerate}
% \item Arista $a: (n_i,n_j)$ tal que $deg(n_i) = 1$ o $deg(n_j) = 1$ 
% %La arista a asignar debe tener al menos uno de sus nodos con grado 1, indicando que es el inicio o final de una parte del grafo.

% \item De no haber aristas disponibles con esas caracter\'isticas, se realiza una asignaci\'on inicial de una arista que cumpla con 2 criterios:
% \begin{itemize}
%     \item $a: (n_i,n_j)$ tal que $deg(n_i) >= 2$ o $deg(n_j) >= 2$ 
%     \item El \'angulo que forma la arista candidata a elegir, junto a otra arista a la que pertenece el nodo, debe pertenecer al rango $]\theta, Max\_Angle]$.
% \end{itemize}

% \item Una arista aleatoria que no pertenezca a una soluci\'on o camino evaluado como de buena calidad
% %. La calidad de un camino se presenta m\'as adelante en esta secci\'on.
% \end{enumerate}
% \end{frame}



\note[itemize]{
    \small
    \item la etapa de construcci\'on de una soluci\'on se compone de 2 partes: La asignaci\'on de la 1ra arista por donde la hormiga comenzar\'a su recorrido y de la elecci\'on de la siguiente arista a recorrer
    \item La asignaci\'on de la primera arista se realiza mediante una heru\'istica que consta de 3 criterios. El enfoque de estos criterios se centra en la exploraci\'on del espacio de b\'usqueda
    \item El primer criterio busca asignar aristas que esten en el borde del grafo, como la arista con uno de sus nodos en A1.
    
    \item De no existir m\'as aristas como las del 1er criterio, se buscan aristas que se encuentren en intersecciones, es decir que tengan uno de sus nodos con grado 2 o superior, y que el angulo que formen las aristas que comparten aquel nodo este en el rango intermedio definido previamente. Esto debido a que en varios tipos de filamentos hay casos en que los filamentos no nacen s\'olo del borde, sino tambi\'en a partir de otros filamentos. Un ejemplo de esto es una de las aristas que contiene al nodo en A2. 
    
    \item De no contar con aristas disponibles seg\'un los criterios previos, se elige una arista al azar, privilegiando las aristas que no son parte de un camino ya evaluado como una soluci\'on factible.
    
   \item Una vez que se le asigna una arista inicial a la hormiga, esta debe elegir que nuevas aristas irá agregando en su camino 

}


\begin{frame}{ACO: Construcci\'on de Soluci\'on (OE 1)}
    \begin{columns}
        \begin{column}{0.65\textwidth}
            \centering
            \includegraphics[scale=0.4]{Pictures/ACO-ant-Constr-choices.png}
        \end{column}
        \begin{column}{0.35\textwidth}
        \small
            \begin{equation}
            P(c_{i} | s^{P}) = \frac
            {\tau_{i} ~ \eta_{i}}
            {\sum\limits_{c_{j}\in N(s^p)}{\tau_{j} ~ \eta_{j} } } %, \forall c_{i} \in N(s^{P}).
            \label{eq:antProbabilities}
            \end{equation}
        \end{column}
    \end{columns}

    \begin{columns}
        \begin{column}{0.35\textwidth}
            \begin{itemize}
                \item Calidad $\sim$ Funci\'on Objetivo 
                %\item Calidad $s^{P}_1$ = $\sum \eta_{i}$
                \item Calidad $s_1$ = $\frac{1}{n}\sum \eta_{i} $
            \end{itemize}
        \end{column}
        \begin{column}{0.65\textwidth}
            \begin{itemize}%\fontsize{9pt}{10}\selectfont
                \item Calidad M\'inima $\geq \frac{Max\_Score}{2}$
                \item Buena Calidad: $Max\_Score$
                \item Calidad Intermedia: $[\frac{Max\_Score}{2}, Max\_Score[$
            \end{itemize}
        \end{column}
    \end{columns}
\end{frame}

\note[itemize]{
    \item Cada una de las posibles aristas vecinas a elegir tiene una probabilidad de ser elegida. Esta probabilidad se define seg\'un la ecuaci\'on a la derecha de la diapositiva y en esta influye la feromona $\tau$ y el valor de una heur\'istica $\eta$ asociada a cada arista.
    \item Esta heur\'istica entrega una puntuaci\'on asociada al \'angulo que forma la arista vecina con la última arista que fue agregada al camino o soluci\'on.
    
    \item a modo de ejemplo, las aristas en rojo representan las aristas vecina de C6, que fue la ùltima en ser añadida hasta ese punto al recorrido de la hormiga, teniendo cada una de las aristas rojas una probabilidad de ser elegida. 

    \item Una vez que la hormiga termina su recorrido, se evalua la soluci\'on mediante el promedio de los valores de eta. Esta evaluaci\'on corresponde a la funci\'on objetivo de la metaheur\'istica ACO y se puede separar en 3 rangos: 
    
    \item Las soluciones que no cumplen con una calidad m\'inima, que se desechan
    \item Las soluciones que se definen como de buena calidad, que son las que se eligen como soluciones candidatas
    \item y las soluciones que requieren de mayor evaluaci\'on dado que con los criterios aplicados durante la construcci\'on no pueden ser desechados ni aceptados.
    
    \item Luego, las soluciones de calidad intermedia y superior pasan a las siguientes etapas.
    
    \item NO LEER - Respira: se busca maximizar la funci\'on objetivo

}

%\subsection{M\'etodo de b\'usqueda no local}
\begin{frame}{B\'usqueda no local: l\'ogicas globales/centralizadas (OE 1)}
    
    \begin{columns}
        \begin{column}{0.4\textwidth}
            \begin{itemize}
                \item Eliminar soluciones candidatas que no aporten informaci\'on nueva
                \item Soluciones parcialmente repetidas
            \end{itemize}
        \end{column}
        \begin{column}{0.2\textwidth}
        \includegraphics[scale=0.5]{Pictures/ant-segments-repetead-sol1.png}
        \end{column}
        \begin{column}{0.2\textwidth}
        \includegraphics[scale=0.5]{Pictures/NoConsenso2.png}
        \end{column}
        \begin{column}{0.2\textwidth}
        \includegraphics[scale=0.5]{Pictures/NoConsenso3.png}
        \vspace{0.5cm}
        \includegraphics[scale=0.5]{Pictures/NoConsenso4.png}
        \end{column}
    \end{columns}
\end{frame}

\note[itemize]{
\small
    \item Una de las etapas siguientes corresponde a la b\'usqueda no local, la que se basa en l\'ogicas globales o centralizadas, a diferencia de la construcci\'on de soluciones de cada hormiga que es miope.
    %\item La b\'usqueda de soluciones de cada hormiga puede llevar a que dos o m\'as hormigas encuentren soluciones que son muy similares, repitiendo informaci\'on.
    \item Esta etapa busca eliminar soluciones duplicadas o muy similares, privilegiando las soluciones que contengan dentro de si a otras. A modo de ejemplo, en naranjo se muestran dos soluciones similares, siendo la soluci\'on de la derecha levemente m\'as larga.
    \item Se elige la soluci\'on m\'as larga para evitar ambiguedades y que en caso de error, la soluci\'on indicada como un filamento sea igual o m\'as extensa en vez de ser corta con respecto a lo que pueda indicar un experto
    \item Este an\'alisis es similar al que puede suceder entre 2 expertos que ante la misma imagen identifiquen distintos filamentos, como se ejemplifica con las im\'agenes de la derecha.
    
}

% \begin{frame}{ACO: Actualizaci\'on de Feromonas (OE 1)}
% \centering
% \includegraphics[scale=0.4]{Pictures/ACO-ant-ferom.png}
% \begin{itemize}
%     
% \end{itemize}
% \end{frame}


%\subsection{Anti-feromonas}
\begin{frame}{ACO: Actualizaci\'on de feronomas, uso de Anti-feromonas SAP (OE 1)}
    \centering
    \includegraphics[scale=0.4]{Pictures/ACO-ant-ferom-penalize.png}
    \begin{columns}
        \begin{column}{0.5\textwidth}
            \begin{itemize}
            \small
                \item Evaluaci\'on de factiblidad de soluciones candidatas $s \in S$
                \item Feronomonas: premiar los caminos de buena calidad para lograr onvergencia sobre un camino \'optimo
                
            \end{itemize}
        \end{column}
        \begin{column}{0.5\textwidth}
            \begin{itemize}
            \small
                \item Cambio: $\tau_0 \longrightarrow \gamma$
                \item $\gamma$ es un factor penaliza los caminos de mala calidad
                \item 2 penalizaciones $\longrightarrow \tau_i = 0$ 
            \end{itemize}
        \end{column}
    \end{columns}
    
\end{frame}

\note[itemize]{
    \item la \'ultima etapa a mencionar corresponde a la actualizaci\'on de feromonas. Al finalizar el recorrido de una hormiga, es necesario actualizar el valor de la feromonas. En el caso de una metaheur\'istica ACO tradicional, se aumenta el valor de tau para cada arista que pertenece a un camino de buena calidad.
    \item Sin embargo, ese enfoque privilegia la convergencia sobre un sola soluci\'on, y a su vez puede ocasionar la perdida de soluciones factibles.
    \item Para evitar eso, se propone el uso de Anti-feromonas, cuyo objetivo es acotar el espacio de b\'usqueda mediante la penalizaci\'on de los caminos o soluciones de mala o baja calidad.
    \item As\'i, el valor de tau que influye en la probabilidad de elecci\'on de una arista disminuye, pudiendo llegar a cero, lo que ocasiona que esa arista no pueda ser elegida por ninguna hormiga que aun este construyendo una soluci\'on.
}

\begin{frame}{Problema de Anti-feromonas SAP (OE 1)}
\centering
\includegraphics[scale=0.4]{Pictures/ACO-ant-constr-penalize.png}
    \begin{itemize}
        \item Problema: penalizaci\'on puede ocasionar perdida de soluciones factibles
        \item Propuesta: Relacionar penalizaci\'on de $\tau_i$ con $c_i \in s^P$
    \end{itemize}
\end{frame}

\note[itemize]{
    \item Sin embargo, el cambio de feromonas a anti-feromonas no soluciona el problema que se puede originar al focalizar la penalizaci\'on solo en las aristas.
    
    \item Una arista con un tau penalizado por pertenecer a uno o más caminos de mala calidad, puede bloquear otros caminos, dado que el valor de tau no guarda relación con las otras aristas que conforman el camino de mala calidad que caus\'o la penalización. Así, las soluciones que pudiesen pasar por esa arista se ven limitadas.
    
    \item A modo de ejemplo, el camino verde que es de mala calidad origina una penalizaci\'on sobre la arista en rojo. Luego, el camino azul, que puede construir una soluci\'on de buena calidad si incorpora la arista en rojo, se ve bloqueado, ya que la arista roja tiene una penalizaci\'on que hace menos probable o derechamente imposible su elecci\'on.
    
    

}

\begin{frame}{Anti-feromonas SAP dependientes del camino previo (OE 1)}
    \centering
    \includegraphics[scale=0.51]{Pictures/ACO-ant-ferom-penalize-seg.png}
\end{frame}

\note[itemize]{
    \item Por ende, se propone un cambio para que la penalizaci\'on de una arista guarde relaci\'on con con el conjunto de aristas que la preceden en el camino
    \item As\'i, solo hormigas que repitan parcial o totalmente l camino verde no podran elegir la arista en rojo, mientras que hormigas que provengan de otro recorrido, como la hormiga del camino azul no se ver\'an afectadas, evitando la perdida de soluciones.
    
    \item El conjunto de aristas predecesoras se denomina segmento.

}

% \begin{frame}{Anti-feromonas sobre un segmento de una sola arista (OE 1)}
%     \centering
%     \includegraphics[height=2in]{Pictures/ant_segments_complex_case_B2.png}
%     \hspace{0.1cm}
%     \includegraphics[height=2in]{Pictures/ant_segments_complex_case_B_blocked.png}
% \begin{itemize}
%     \item Segmentos de una sola arista ocasionan mismo problema que se intenta resolver
% \end{itemize}
% \end{frame}

% \begin{frame}{Anti-feromonas sobre un segmento (OE 1)}
% \centering
%     \includegraphics[scale=0.5]{Pictures/ant_segments_complex_case_B2_extended.png}
%     \begin{itemize}
%         \item Se extiende el segmento unitario con el segmento que lo precede
%     \end{itemize}
% \end{frame}

%\subsection{Criterios para la Actualizaci\'on de Anti-feromonas}
\begin{frame}{Criterios de Anti-feromonas sobre soluciones de calidad intermedia (OE 1)}
\begin{enumerate}
    
    \item Curvatura de una soluci\'on
    \item Magnitud de desplazamiento entre segmentos
    \item Criterio espec\'ifico para neuronas
\end{enumerate}
\end{frame}

\note[itemize]{
    \item Los 2 primeros criterios se relacionan a la rigidez que presenta un filamento, el que varía dependiendo de la estructura observada. 
}

\begin{frame}{Curvatura de una soluci\'on (OE 1)}
\centering
    \includegraphics[scale=0.5]{Pictures/ant_curvature_case.png}
    \begin{itemize}
        \item El \'angulo entre la proyecci\'on de $\Vec{p}$ y $\Vec{q}$ debe ser menor a $\theta \times Max\_Axial\_Displacement$ para no descartar la soluci\'on.
    \end{itemize}
\end{frame}

\note[itemize]{
\item una soluci\'on como la destacada en color naranja se puede calificar como una soluci\'on infactible debido a su pronunciada curvatura, lo que no es normal de observar en los filamentos. 
\item As\'i, el c\'alculo de esta curvatura corresponde al \'angulo formado entre el vector q y la proyecci\'on del vector p. Este \'angulo debe respetar el umbral que define la multiplicaci\'on de los par\'ametros theta y Max axial displacement
}

\begin{frame}{Magnitud de desplazamiento entre segmentos (OE 1)}
    \centering
    \includegraphics[height=2in]{Pictures/ant_segmentMagnitude_case.png}
    \hspace{0.2cm}
    \includegraphics[height=2in]{Pictures/ant_segmentMagnitude_case_2.png}
    \begin{itemize}
        %\item El criterio de curvatura puede no ser suficiente por si mismo
        \item Comportamiento esperado de la rigidez permite descartar soluciones con cambios demasiado pronunciados entre sus segmentos
    \end{itemize}
\end{frame}

\note[itemize]{
\item 
}

%%\subsection{Extracci\'on de informaci\'on para individualizar filamentos}
\begin{frame}{Criterio espec\'ifico para neuronas (OE 1)}
    \begin{itemize}
        \item Se debe validar que los filamentos de una neurona parten del {\bf soma}
        \item  Extracci\'on de informaci\'on: caracter\'isticas geom\'etricas, topol\'ogicas y espaciales
    \end{itemize}
    \begin{columns}
        \begin{column}{0.55\textwidth}
        \centering
        \includegraphics[height=1.7in]{Pictures/Porta183-somaEdges-example2.png}
        \end{column}
        \begin{column}{0.45\textwidth}
            \begin{equation}
                \tau_{ij} \leftarrow
                    \begin{cases}
                     \tau_{ij} \cdot \gamma \text{ si } deg(v^{a}_{n}) = 1,  \\[3ex]
                    
                    \text{0 si } \tau_{ij} \leq 0.25, \\[3ex]
                    \tau_{ij} \quad \text{en otro caso}.
                    \end{cases}
            \end{equation}
        \end{column}
    \end{columns}
    
\end{frame}

% \begin{frame}{Extracci\'on de informaci\'on para individualizar filamentos}
%      \begin{figure*}[h!]
%     \centering
%     \begin{subfigure}[t]{0.48\textwidth}
%         \centering
%         %\includegraphics[height=1.5in]{Pictures/50-ROIs-Spinning-Marchantia-somaEdges.png}
%     \end{subfigure}%
%     ~ \hspace{0.1cm}
%     \begin{subfigure}[t]{0.48\textwidth}
%     \centering
%         \includegraphics[height=1.5in]{Pictures/Porta18-3a1-somaEdges.png}
%     \end{subfigure}
%     \end{figure*}
%     \begin{itemize}
%         \item Caracter\'isticas geom\'etricas, topol\'ogicas y espaciales
%     \end{itemize}
% \end{frame}
\section{Resultados}



\begin{frame}{Metodolog\'ia}
\begin{itemize}
    \item Par\'ametros principales de las configuraciones utilizadas:
        \begin{itemize}
            \item $\theta$
            \item $Max\_Axial\_Displacement$
            \item Uso de 1 o 3 pasos de la heur\'istica de asignaci\'on inicial
        \end{itemize}
    
    \item 12 Im\'agenes: 2 Sint\'eticas, 7 de Microt\'ubulos en {\it Arabidopsis Marchantia} y 3 de neuronas de rat\'on
    \begin{itemize}
        \item Caso evaluado por 2 expertos
    \end{itemize}
    
    %\item M\'etricas y Medidas:  VI, \'Indice Rand, \'Indice Jaccard
    \item Evaluaci\'on de:
        \begin{itemize}
            \item {\it Precision} y {\it Recall}
            \item N$\degree$ filamentos propuestos y correctos vs {\it ground truth}
        \end{itemize}
\end{itemize}
\end{frame}

\note[itemize]{
    \item Antes de discutir los resultados obtenidos, es necesario explicar los par\'ametros considerados. Dentro de los par\'ametros disponibles para personalizar el algoritmo propuesto, los principales par\'ametros utilizados en las pruebas fueron el umbral theta, el par\'ametro max axial displacement y el uso parcial o total de la heurística de asignaci\'on inicial debido a que estos par\'ametros fueron los que mostraron mayor impacto en los resultados
    \item Las pruebas considerar un total de 12 im\'agenes entre sint\'eticas y reales, con una imagen de microt\'ubulo de planta siendo evaluada por 2 expertos.
    \item La evaluaci\'on considera el n\'umero de filamentos propuestos as\'i como los filamentos correctos que indica el algoritmo propuesto, con respecto a lo indicado por un experto en cada caso. Esta informaci\'on se complementa con las medidas de precision y recall.
    
    %\item (SKIP) N$\degree$ Filamentos Propuestos y Correctos vs {\it Ground Truth}
    
    \item El resultado de la individualización de filamentos para cada imagen, consiste en el promedio de 10 iteraciones del algoritmo propuesto. Debido a que es un algoritmo que incluye c\'alculos de probabilidad, en cada iteraci\'on se cambia la semilla con la finalidad de explorar el espacio de b\'usqueda desde distintas partes.
}

% \begin{frame}{Im\'agenes Sint\'eticas}

%     \begin{figure*}[h]
%         \centering
%         \begin{subfigure}[t]{0.31\textwidth}
%             \centering
%             \includegraphics[height=1.3in]{Pictures/Synth-QuantitativeIFS-Fig7_groundTruth.png}
%         \end{subfigure}
%         ~ %\hspace{0.1cm}
%         \begin{subfigure}[t]{0.31\textwidth}
%             \centering
%             \includegraphics[height=1.3in]{Pictures/QFS7-DeFiNeExactMatch-30.png}
%         \end{subfigure}
%         ~
%         \begin{subfigure}[t]{0.31\textwidth}
%             \centering
%             \includegraphics[height=1.3in]{Pictures/Synth-QuantitativeIFS-Fig7-phil-s1271-v056-exactMatch-antLabeled.png}
%         \end{subfigure}
%     \end{figure*}
    
% \resizebox{\textwidth}{!}{
%         %\footnotesize
%         \begin{tabular}{|c|c|c|c|c|c|c|c|c|c|c|c|c|}
%         \hline
%             Config & Iters & P & P* & R & R* & F1 & F1* & C/P & C/P* & C/GT & C/GT* & T[s]\\ \hline
%              DeFiNe 30\textdegree  & 1 & 0.72 & - & 0.47 & - & 0.57 & - & 4/6 & - & 4/6 & - & 2.3 \\
%              DeFiNe 60\textdegree & 1 &0.63 & - & 0.58 & - & 0.60 & - & 3/5 & - & 3/6 &- & 2.3\\
%             AP MT-P & 5 & 0.51 & 0.57 & 0.32 & 0.57 & 0.39 & 0.57 & 3/6.2 & 3/5 & 3/6 & 3/6 & 0.3\\
%             %Mejor Iteraci\'on P1 & 0.5714 & 0.5714 & 0.5714 & 3/5 &  & 0.3135 \\
%             AP S1 & 5 & 0.68 & 0.87 &0.57 & 1 & 0.62 & 0.93 & 4/5.8 & 4/5 & 4/6 & 4/6 & 0.2\\
%             % {\bf Mejor Iteraci\'on P2} & 0.875 & 1 & 0.9333 & 4/5 & 4/6 & 0.3073\\
%              \hline
%         \end{tabular}
%     }
% \end{frame}

\begin{frame}{Im\'agenes Sint\'eticas (OE 4)}
    \vspace{-0.6cm}
    \begin{columns}
        \begin{column}{0.31\textwidth}
        
            \begin{figure}
                \centering
                \includegraphics[height=1.3in]{Pictures/define-weighted-4-groundTruth.png}
                \caption{Ground Truth}
            \end{figure}
        \end{column}
        \begin{column}{0.31\textwidth}
        %\vspace{-1cm}
            \begin{figure}
                \centering
                \includegraphics[height=1.3in]{Pictures/defineFig1b-DeFiNeExactMatch-30.png}
                \caption{DeFiNe 30\textdegree}
            \end{figure}
        \end{column}
        \begin{column}{0.31\textwidth}
        %\vspace{-1cm}
            \begin{figure}
                \centering
                \includegraphics[height=1.3in]{Pictures/define-weighted-4-phil-s3389-v056-exactMatch-antLabeled.png}
                \caption{AP S2}
            \end{figure}
        \end{column}
    \end{columns}

    \vspace{-0.3cm}
    \begin{table}[h]
    \resizebox{0.85\textwidth}{!}{
        \begin{tabular}{|c|c|c|c|c|c|c|c|c|}
        \hline
        Config & Iters & P & R & C/P & C/P* & C/GT & C/GT* & T[s] \\ \hline
        DeFiNe 30\textdegree & 1 & 0.33 & 0.18 & 2/11 & - & {\bf 2/5} & - & 2.8 \\
        DeFiNe 60\textdegree & 1& 0.23 & 0.25 & 2/7 & - & 2/5 & - & 3.6\\
        AP MtP & 10 & 0.34 & 0.27 & 3/9.2 & 3/9 & 3/6 & 3/6 & 0.3\\
        AP S2 & 10 & 0.32 & 0.25 & 4/8.6 & 4/9 & {\bf 4/6} & 4/6 & 0.3\\
            \hline
        \end{tabular}
    }
    \caption{(*) indica el mejor resultado de las 10 iteraciones.}
    \end{table}
\end{frame}

\note[itemize]{
\scriptsize
  \item A modo de ejemplo para las imagenes sint\'eticas evaluadas, se presenta una imagen sint\'etica que contiene 5 filamentos. El grafo que representa estos 5 filamentos esta compuesto por 17 aristas.
  
  %\item esta imagen sintetica se obtiene del paper de Define

 \item Para DeFiNe, se consideran 2 configuraciones id\'enticas basadas en los mejores resultados obtenidos por esa herramienta de acuerdo a sus autores, y s\'olo se varia el \'angulo de umbral de BFS, mientras que con el algoritmo propuesto se utilizaron 2 configuraciones, una predeterminada para microt\'ubulos de planta, definida como MtP y otra personalizada definida como S2
 
 \item Tanto para DeFiNe como para el algoritmo propuesto, los resultados de precision y recall no son muy elevados. Sin embargo se observa que ambas configuraciones del algoritmo propuesto tienen un comportamiento similar.
 
 \item En particular, en la 5ta columna, C vs P, que representa los filamentos correctamente identificados versus el n\'umero de filamentos propuestos, la configurac\'on predeterminada para microt\'ubulo de planta logra identificar correctametne 3 filamentos de los 9 que propone, mientras que la configuraci\'on S2, que tiene ajustes particulares para esta imagen, obtiene una identificaci\'on correcta adicional, y a la vez reduce el promedio de filamentos propuestos.
 
 \item Una de las modificaciones de la configuraci\'on S2 con respecto a la de microt\'ubulos es que S2 solo utiliza el primer criterio en la heur\'istica de asignaci\'on inicial de aristas, ya que este caso sint\'etico no contiene filamentos que nacen a partir de otros.
 %\item La idea principal es que el n\'umero de filamentos correctamente identificados sea sino identico, lo m\'as cercano al n\'umero de filamentos propuestos, ya que 
 
 
 \item Es necesario mencionar que en las columnas C vs GT, que representan los filamentos identificados correctamente con respecto a lo definido por un experto, se consideran 6 filamentos como ground truth para el algoritmo propuesto  debido a que la herramienta que genera el grafo para el algoritmo propuesto gener\'o una arista aislada en la parte superior, que realmente corresponde a la continuaci\'on de una arista y no a otra por si sola. 
 
 \item En la visualizaci\'on de los filamentos identificados por el algoritmo propuesto, a la derecha de la presentaci\'on, se observa que los filamentos no identificados corresponden a los que presentan una mayor curvatura.
 Es probable que con una configuraci\'on m\'as afinada, el algoritmo propuesto pueda encontrar los 2 filamentos faltantes, ya que ese comportamiento se aleja un poco de lo observado en los filamentos incluidos en este trabajo.
}

\begin{frame}{Muestra MT-A Microt\'ubulo (OE 4)}
\vspace{-1cm}
    \begin{columns}
        \begin{column}{0.22\textwidth}
            \begin{figure}
                \centering
                \includegraphics[height=1.3in]{Pictures/SPINNING-DISK-MARCHANTIA-rois-unlabeled.png}
                \caption{Imagen Original}
            \end{figure}
        \end{column}
        \begin{column}{0.22\textwidth}
            \begin{figure}
                \centering
                \includegraphics[height=1.3in]{Pictures/50-ROIs-Spinning-Marchantia-solved-rot-unlabeled.png}
                \caption{Ground Truth}
            \end{figure}
        \end{column}
        \begin{column}{0.22\textwidth}
            \begin{figure}
                \centering
                \includegraphics[height=1.3in]{Pictures/50-ROIs-Spinning-Marchantia-DeFiNeExactMatch-30.png}
                \caption{DeFiNe 30\textdegree}
            \end{figure}
        \end{column}
        \begin{column}{0.22\textwidth}
            \begin{figure}
                \centering
                \includegraphics[height=1.3in]{Pictures/50-ROIs-Spinning-Marchantia-phil-s10-v05-nobg-antLabeled.png}
                \caption{AP S2}
            \end{figure}
        \end{column}
    \end{columns}

    \begin{table}[h]
    \resizebox{0.85\textwidth}{!}{
        \begin{tabular}{|c|c|c|c|c|c|c|c|c|c|c|}
        \hline
              Config & Iters & P & R & C/P & C/P* & C/GT & C/GT* & T[s] \\ \hline
             DeFiNe 30\textdegree & 1 & 0.65 & 0.45 & 8/16 & - & {\bf 8/12} & - & 4.1 \\
             DeFiNe 60\textdegree & 1& 0.28 & 0.21 & 3/12 &- & 3/12 & - & 3.6\\
            AP MT-P & 10 & 0.44 & 0.34 & 7.2/12.4 & 8/13 & 7.2/12 & {\bf 8/12} & 0.3\\
             \hline
        \end{tabular}
        }
    \caption{Grafo de 29 aristas}
    \end{table}
\end{frame}

\note[itemize]{
  \item grafo de 29 aristas
}

\begin{frame}{Neuronas de rat\'on (OE 4)}
%\vspace{-1cm}
    \resizebox{\textwidth}{!}{
        \begin{tabular}{|c|c|c|c|c|r|c|}
        \hline
             Muestra & Algoritmo & Fil. Propuestos & {\it GT} &\% Asignaci\'on & Tiempo[s] & N\textdegree~ Aristas\\
             \hline
             \multirow{3}{*}{N1}& DeFiNe 30\textdegree & 246 & \multirow{3}{*}{24} &100 & 1514.6 & \multirow{3}{*}{414} \\
                    & DeFiNe 60\textdegree & 192 && 100 & 15573.7 &\\
                    & AP N Promedio & 59 && 53.7 & 32.5 &\\ \hline
            \multirow{3}{*}{N2}& DeFiNe 30\textdegree & 113 & \multirow{3}{*}{29} & 100 & 82.2 & \multirow{3}{*}{161}\\
                    & DeFiNe 60\textdegree & 85 && 100 & 456.4 &\\
                    & AP N Promedio & 34.8 && 59.3 & 4.9 &\\ \hline
            N3 & AP N Promedio & 17.4 & 14 & 57.8 & 4.2 & 67\\ \hline
        \end{tabular}
    }
    \vspace{0.5cm}
    \begin{itemize}
        \item \% Asignaci\'on debe estar relacionado a la calidad de la informaci\'on, evitando asignar s\'olo por cumplir con el modelo
        \item N\textdegree~ Fil. Prop. refleja acotamiento del espacio de b\'usqueda
    \end{itemize}
\end{frame}

\note[itemize]{
\small
  \item En las neuronas de raton, las pruebas ejecutadas con DeFiNe presentan cantidades de filamentos propuestos 2 a 6 veces mayores que el n\'umero de filamentos individualizados por un experto. Este comportamiento puede atribuirse a la obligaci\'on que DeFiNe tiene de asignar todas las aristas al menos a un filamento. En comparaci\'on, el algoritmo propuesto al tener mayor flexibilidad, asigna en promedio un 57\% de las aristas a filamentos.
  
  \item Adicionalmente, los tiempos de ejecuci\'on de las pruebas con DeFiNe son sustancialmente mayores, encontr\'andose en el rango de los minutos a las 4 horas, dependiendo de los par\'ametros utilizados. Lo anterior puede asociarse al n\'umero de aristas que las muestras de neurona tienen en su respectivo grafo
  
  \item En cuanto al algoritmo propuesto, a pesar de ejecutar todas las evaluaciones en tiempos menores a 40 segundos, las individualizaciones correctas son bajas, mientras que el n\'umero de filamentos propuestos se acerca a la cantidad de filamentos individualizados manualmente por un experto, exceptuando el caso de la Muestra N1.
  
  \item  Es posible asociar un comportamiento razonable del n\'umero de filamentos propuestos con la restricci\'on especial para neuronas considerada dentro de las penalizaci\'on de anti-feromonas, lo que permite mantener acotado el n\'umero de filamentos propuestos mediante el descarte de soluciones infactibles.
}

\begin{frame}{Muestra N3 de neurona de rat\'on (OE 4)}
    \vspace{-0.2cm}
        \begin{columns}
        \begin{column}{0.45\textwidth}
            \begin{figure}
                \centering
                \includegraphics[height=1.35in]{Pictures/Porta18-3a1-phil-s10-v056-exactMatch-antLabeled.png}
                \caption{Muestra N3 Calce Exacto}
            \end{figure}
        \end{column}
        \begin{column}{0.45\textwidth}
            \begin{figure}
                \centering
                \includegraphics[height=1.35in]{Pictures/Porta18-3a1-phil-s10-v056-overmatches-3-antLabeled.png}
                \caption{Muestra N3 con sub/sobre asignaci\'on}
            \end{figure}
        \end{column}
    \end{columns}
    
    \begin{table}[h]
    \resizebox{0.65\textwidth}{!}{
        \begin{tabular}{|c|c|c|c|c|c|c|c|}
        \hline
              Muestra & C/Prop. & C/Prop.* & C/GT & C/GT* & F. Sub/Sobre asig. \\ \hline
            N1 & 1.6/59 & 3/66 & 1.6/24 & 3/24 & 1.4\\
            N2 & 4.6/34.8 & 6/34 & 4.6/29 & 6/29 & {\bf 8} \\
            N3 & 2/17.4 & 2/17 & 2/14 & 2/14 & {\bf 7.8}\\
            \hline
        \end{tabular}
        }
        \caption{(*) indica el mejor resultado de las 10 iteraciones}
    \end{table}
\end{frame}

\note[itemize]{
  \item En particular, al estudiar los filamentos propuestos por el algoritmo para el caso de neuronas, se observa que a\'un cuando los calces exactos son bajos, existen varios filamentos propuestos que se encuentran en un rango de 1 a 3 aristas faltantes o sobrantes, con respecto a un filamento correcto.
  
  \item as\'i, al incluir los filamentos propuestos con una diferencia de hasta 3 aristas, lo cual se refleja en las \'ultima columna de filamentos con sub o sobre asignaci\'on, se pueden encontrar varios filamentos candidatos descartados que se encuentran cercanos a lo definido por un experto, lo que indica para las neuronas puede ser necesario incluir un an\'alisis adicional durante la construcci\'on de soluciones mediante las hormigas.
  
  \item Lo anterior mejorar\'ia la cantidad de filamentos correctos, manteniendo el comportamiento del n\'umero de filamentos propuestos 
}

\begin{frame}{Resultados Generales (OE 4)}
\resizebox{0.85\textwidth}{!}{
    \begin{tabular}{|c|c|c|c|c|}
        \hline
        Figura & Config. & Fil. Correctos & Fil. Propuestos & $p$-value  \\ \hline
         4.1 & S1  & 6 & 5.9 & {\bf 1} \\
         4.2 & S2 & 5 & 8.5 & 0.003\\
         4.3 & MtP & 12 & 12.4 & {\bf 0.125}\\
         4.4 & MtP & 12 & 15 & 0\\
         4.5 & MtP & 5 & 5 & {\bf 1}\\
         4.6 & N & 24 & 60.2 & 0\\
         4.7 & N & 29 & 34.3 & 0\\
         4.8 & N & 14 & 17.6 & 0.003\\ 
         5.3b & MtP & 19 & 21 & 0.001\\
         5.3c & MtP & 23 & 29.4 & 0 \\
         5.9a & MtP & 13 & 14 & 0.001\\
         5.10a & MtP & 12 & 13.1 & 0\\
         \hline
    \end{tabular}
}
\end{frame}

\note[itemize]{
    
    \item En general, el hecho que el algoritmo propuesto no tenga diferencias estad\'isticamente significativas en 3 de las pruebas, as\'i como que presente un n\'umero de filamentos propuestos cercano a lo indicado por un experto en la mayor\'ia de los casos representa la capacidad del algoritmo de acotar el espacio de b\'usqueda de forma eficiente
    
    \item Es relevante destacar que la configuraci\'on predefinida para neuronas o para microt\'ubulos de planta no ha sido ajustada o sintonizada. Es probable que con una sintonización de par\'ametros, los resultados mejoren.
    
    \item SKIP Se defini\'o que La hip\'otesis nula (H0) fuese la no existencia de diferencia estad\'isticamente significativa entre el n\'umero de filamentos individualizados por el algoritmo propuesto y lo definido por el experto que realiz\'o la individualizaci\'on manual respectiva. En el caso de que el valor $p$ o {\it p-value} sea menor al 5\%, se rechaza H0. 
}


\section{Conclusiones}
\begin{frame}{Conclusiones}
    %cumplimiento de objetivos
% a partir de un grafo con pesos que representa una red de filamentos, cumpliendo con lo indicado como primer objetivo espec\'ifico. En cuanto al cumplimiento de los dem\'as objetivos espec\'ificos, se tiene:
\begin{itemize}
    \item El algoritmo propuesto permite resolver un modelo de optimizaci\'on para individualizar filamentos

    \item Se gener\'o una implementaci\'on que resuelve el modelo de optimizaci\'on presentado, considerando casos de solapamiento y/o cruce
    
    \item Se incorporan diversas caracter\'isticas en el algoritmo propuesto
    %\item  {\bf Identificar la ponderaci\'on de propiedades que entregue mejores resultados para grafos que representen una neurona, una bacteria y una c\'elula eucariota de planta:}: Se realiza una ponderaci\'on de las propiedades utilizadas en el algoritmo propuesto en la secci\'on ..., la que var\'ia dependiendo del tipo de c\'elula y se encuentra asociado a la distribuci\'on de las caracter\'isticas en las distintas etapas de la metaheur\'istica ACO.
    
    \item Se eval\'ua el algoritmo propuesto con respecto a DeFiNe
  %  \item {\bf Evaluar t\'ecnicas que usan s\'olo herramientas de visi\'on por computador, basadas en poblaci\'on de p\'ixeles y otras que utiliza un m\'etodo derivado de contornos activos:} Las t\'ecnicas indicadas se encuentran descritas en el cap\'itulo..., y se enfocan en la extracci\'on de informaci\'on geom\'etrica y/o topol\'ogica. Sin embargo no realizan individualizaci\'on de filamentos, por lo que la evaluaci\'on fue dirigida hacia la investigaci\'on que da lugar a DeFiNe, que si realiza este procedimiento, siendo evaluada en el cap\'itulo ...
  \item Se obtiene un comportamiento estable del algoritmo propuesto
  %\item La flexibilidad del algoritmo propuesto permite 
\end{itemize}
\end{frame}

\note[itemize]{
\item En general, el algoritmo propuesto permite resolver un modelo de optimizaci\'on para individualizar filamentos a partir de un grafo con pesos que representa una red de filamentos, cumpliendo con lo indicado como primer objetivo espec\'ifico. En cuanto al cumplimiento de los dem\'as objetivos espec\'ificos, se tiene:
    \item  El algoritmo propuesto cumple con el 2do objetivo espec\'ifico, de acuerdo a lo presentado mediante el modelo de optmiizaci\'on que utiliza la meta ACO, as\'i como de los resultados obtenidos
    
    \item Se realiza una ponderaci\'on de las propiedades utilizadas en el algoritmo propuesto , la que var\'ia dependiendo del tipo de c\'elula y se encuentra asociado a la distribuci\'on de las caracter\'isticas en las distintas etapas de la metaheur\'istica ACO.
    
    \item  Las t\'ecnicas indicadas se encuentran descritas en el cap\'itulo \ref{chap:stateoftheart}, y se enfocan en la extracci\'on de informaci\'on geom\'etrica y/o topol\'ogica. Sin embargo no realizan individualizaci\'on de filamentos, por lo que la evaluaci\'on fue dirigida hacia la investigaci\'on que da lugar a DeFiNe, que si realiza este procedimiento, siendo evaluada en el cap\'itulo result.
}

% \begin{frame}{Conclusiones}
%     \item Como cotinuaci\'on de esta investigaci\'on se encuentra la opci\'on de intensificar la extracci\'on de informaci\'on topol\'ogica y espacial que permita realizar la asociaci\'on de secciones importantes de los filamentos o elementos alrededor de estos en la c\'elula observada, con el fin de seguir a\~nadiendo informaci\'on al modelo. Tambi\'en es posible vislumbrar una extensi\'on de este trabajo en la identificaci\'on de las interacciones entre los filamentos individualizados. Lo anterior se refleja en microt\'ubulos como la identificaci\'on de interacciones tipo cat\'astrofe, zippering o nucleaci\'on, mientras que para una neurona ser\'ia la clasificaci\'on autom\'atica de los filamentos en axon o dendritas primarias o secundarias.
% \end{frame}

\begin{frame}{Agradecimientos}
    Esta tesis ha sido parcialmente financiada por el Fondo Nacional de Desarrollo Cient\'ifico y Tecnol\'ogico (FONDECYT 1180906 y 1190806) y la ICM (P09-015-F)
\end{frame}

% TITLE PAGE	

{
\setbeamertemplate{headline}{} %define local, empty header for title page
\setbeamertemplate{footline}{} %define local, empty footer for title page
\maketitle
}
\addtocounter{framenumber}{-1} % We don't count the title page

\addtocontents{toc}{\protect\setcounter{tocdepth}{0}}
\section{Anexo}
\begin{frame}{VI, \'Indice Rand e \'Indice Jaccard}
\resizebox{\textwidth}{!}{
    \begin{tabular}{|c|c|c|c|c|c|}
        \hline
        Figura & Config. de Par\'ametros & VI Max & VI & Rand & Jaccard \\ \hline
         4.1 & MT-P  & 2.3978 & 1.6360 & 0.7646 & 0.2485 \\
         4.1 & S1  & 2.3978 & 0.7091 & 0.8637 & 0.4750  \\
         4.2 & MT-P & 3.0445 & 2.2296 & 0.7276 & 0.1806 \\
         4.2 & S2 & 3.0445 & 2.5878 & 0.7276 & 0.1673  \\
         4.3 & MT-P & 3.4965 & 2.1656 & 0.8658 & 0.2407 \\
         4.4 & MT-P & 3.7612 & 2.6285 & 0.8683 & 0.2488  \\
         4.5 & MT-P & 1.9459 & 0.4286 & 0.8929 & 0.40 \\
         4.6 & N & 6.0258 & 1.7950 & 0.8864 & 0.1389 \\
         4.7 & N & 5.0814 & 3.7256 & 0.8775 & 0.0703 \\
         4.8 & N & 4.2046 & 1.0060 & 0.8794 & 0.2157 \\ \hline
    \end{tabular}
    %\caption{VI $\in [0, VI\_Max]$, Rand y Jaccard $\in [0, 1]$ }
}
\end{frame}

\end{document}